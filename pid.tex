\documentclass{article}
\usepackage[utf8]{inputenc}
\usepackage{xeCJK}
\usepackage{booktabs}
\usepackage{longtable}
\usepackage{array}
\usepackage{tikz}
\usetikzlibrary{arrows.meta}
\usepackage{xeCJK}
\usepackage{CJKutf8}
\setCJKmainfont{Noto Serif CJK TC}    % 思源宋體繁體版(推薦)
\setCJKsansfont{Microsoft JhengHei}   % 微軟正黑體
\setCJKmonofont{Noto Sans Mono CJK TC}
\title{PID 控制理論發展時間線}
\author{}
\date{}

\begin{document}
\maketitle

以下是 \textbf{PID 控制理論發展時間線},方便記憶各個重要里程碑:

\section{PID 控制發展時間線}

\begin{longtable}{p{2cm} p{4cm} p{8cm}}
\toprule
\textbf{年代} & \textbf{事件} & \textbf{說明} \\
\midrule
\textbf{1890s} & 比例控制(P 控制)應用 & 用於蒸汽機調速器、溫度控制等,但不能消除穩態誤差 \\
\addlinespace
\textbf{1922} & \textbf{Nicolas Minorsky} 提出 PID & 在美國海軍自動舵研究中,首次提出比例 + 積分 + 微分的控制方法 \\
\addlinespace
\textbf{1930--1940s} & 工業應用雛形 & 機械與氣動控制器引入 PID 三項設計 \\
\addlinespace
\textbf{1942} & Taylor Instrument 推出商用 PID & 氣動式 PID 控制器進入工業市場 \\
\addlinespace
\textbf{1940s} & \textbf{Ziegler--Nichols} 發表整定法 & 提出經驗公式,用臨界增益與臨界週期整定 PID,快速推廣 \\
\addlinespace
\textbf{1970s} & 數位化 PID 興起 & 微處理器可直接實作 PID 演算法,取代機械與氣動式 \\
\addlinespace
\textbf{1980s--至今} & 多種改良型 PID & 增加前饋補償、自整定、模糊 PID、適應控制等,應用於航太、機器人、製程控制等 \\
\bottomrule
\end{longtable}

\section{圖示化時間軸}

\begin{center}
\begin{tikzpicture}[scale=1.2]
% Timeline arrow
\draw[-{Stealth[length=5mm]}] (0,0) -- (12,0);

% Year markers
\draw (1,0) -- (1,-0.3) node[below] {1890s};
\draw (3,0) -- (3,-0.3) node[below] {1922};
\draw (6,0) -- (6,-0.3) node[below] {1940s};
\draw (9,0) -- (9,-0.3) node[below] {1970s};
\draw (11.5,0) -- (11.5,-0.3) node[below] {現代};

% Event labels above timeline
\node[above, text width=1.5cm, align=center] at (1,0.5) {P控制};
\node[above, text width=2cm, align=center] at (3,0.5) {Minorsky};
\node[above, text width=2.5cm, align=center] at (6,0.5) {Z-N整定 \& 商用PID};
\node[above, text width=2cm, align=center] at (9,0.5) {數位化PID};
\node[above, text width=2cm, align=center] at (11.5,0.5) {改良型PID};

% Connecting lines
\draw[dashed] (1,0.3) -- (1,0);
\draw[dashed] (3,0.3) -- (3,0);
\draw[dashed] (6,0.3) -- (6,0);
\draw[dashed] (9,0.3) -- (9,0);
\draw[dashed] (11.5,0.3) -- (11.5,0);

\end{tikzpicture}
\end{center}

\vspace{1cm}

\end{document}