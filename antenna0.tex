\documentclass{article}
\usepackage{tikz}
\usetikzlibrary{arrows.meta, positioning, shapes.geometric}
\usepackage[utf8]{inputenc}
\usepackage[UTF8]{ctex}
\usepackage{pgfplots}
\pgfplotsset{compat=1.16}
\usetikzlibrary{calc}
\usepackage{amsmath}
\usepackage{amssymb}
\usepackage{array}
\usepackage{booktabs}
\usepackage{physics}
\usepackage{siunitx}
\usepackage{enumitem}
\usepackage{subcaption}
\usepackage[unicode]{hyperref}
\pdfstringdefDisableCommands{%
  \def\pi{pi}%
  \def\Omega{Omega}%
  \def\hat#1{#1}%
  \def\dfrac#1#2{#1/#2}%
  \def\int{\text{∫}}%
}
\usepackage{xeCJK}
\usepackage{bm}
\usepackage{CJKutf8}
\setCJKmainfont{Noto Serif CJK TC}    % 思源宋體繁體版(推薦)
\setCJKsansfont{Microsoft JhengHei}   % 微軟正黑體
\setCJKmonofont{Noto Sans Mono CJK TC}
% \setCJKmainfont{WenQuanYi Zen Hei}
% % 或者使用其他可用字體:
% % \setCJKmainfont{AR PL Mingti2L Big5}
% % \setCJKmainfont{Noto Sans CJK TC}

% % 設置全形符號支持
% \setCJKsansfont{WenQuanYi Zen Hei}
% \setCJKmonofont{WenQuanYi Zen Hei}

\title{design of planar antenna}
\author{carlos ma}
\date{\today}
\usepackage[margin=1in]{geometry}
% \usepackage{tikz}
\usetikzlibrary{arrows.meta,positioning}
\usepackage{caption}
\begin{document}
\maketitle
\section*{(1) 勢函數的波動方程(Lorenz 規範)}

由勢的定義
\[
\mathbf{B}=\nabla\times\mathbf{A},\qquad
\mathbf{E}=-\nabla\Phi-\frac{\partial\mathbf{A}}{\partial t},
\]
配合 Lorenz 規範
\[
\nabla\cdot\mathbf{A}+\mu\varepsilon\frac{\partial\Phi}{\partial t}=0,
\]
可得含源波動方程:
\begin{align}
\nabla^{2}\Phi - \mu\varepsilon \frac{\partial^{2}\Phi}{\partial t^{2}}
&= -\frac{\rho}{\varepsilon}, \label{eq:Phi}\\[6pt]
\nabla^{2}\mathbf{A} - \mu\varepsilon \frac{\partial^{2}\mathbf{A}}{\partial t^{2}}
&= -\mu\,\mathbf{J}. \label{eq:A}
\end{align}

---

\section*{(2) 電場的波動方程}

從法拉第定律取旋度:
\[
\nabla\times(\nabla\times\mathbf{E})
=-\frac{\partial}{\partial t}(\nabla\times\mathbf{B}),
\]
利用向量恆等式與高斯定律、安培–麥斯威爾定律,得
\begin{equation}
\nabla^2 \mathbf{E}-\mu\varepsilon\,\frac{\partial^2 \mathbf{E}}{\partial t^2}
=\nabla\!\left(\frac{\rho}{\varepsilon}\right)+\mu\,\frac{\partial \mathbf{J}}{\partial t}.
\label{eq:Ewave}
\end{equation}

---

\section*{(3) 磁場的波動方程}

由安培–麥斯威爾定律取旋度並代入 \(\nabla\times\mathbf{E}=-\partial_t\mathbf{B}\):
\begin{equation}
\nabla^2 \mathbf{B}-\mu\varepsilon\,\frac{\partial^2 \mathbf{B}}{\partial t^2}
=-\,\mu\,\nabla\times\mathbf{J}.
\label{eq:Bwave}
\end{equation}

---

\section*{(4) 無源情況}

若 \(\rho=0,\ \mathbf{J}=0\),則 \eqref{eq:Ewave} 與 \eqref{eq:Bwave} 退化為齊次波動方程:
\[
\nabla^2 \mathbf{E}-\mu\varepsilon \frac{\partial^2 \mathbf{E}}{\partial t^2}=0,\qquad
\nabla^2 \mathbf{B}-\mu\varepsilon \frac{\partial^2 \mathbf{B}}{\partial t^2}=0,
\]
對應到自由空間的電磁波傳播,波速為 \(c=1/\sqrt{\mu\varepsilon}\)。

---

\section*{(5) 比較總結}

\renewcommand{\arraystretch}{1.4}
\begin{tabular}{|>{\raggedright}m{3.5cm}|>{\raggedright}m{5cm}|>{\raggedright\arraybackslash}m{5cm}|}
\hline
 & \textbf{勢函數波動方程} & \textbf{場的波動方程} \\
\hline
形式 &
\(\nabla^2\Phi-\mu\varepsilon\,\partial_t^2\Phi=-\rho/\varepsilon\) \newline
\(\nabla^2\mathbf{A}-\mu\varepsilon\,\partial_t^2\mathbf{A}=-\mu\mathbf{J}\) &
\(\nabla^2 \mathbf{E}-\mu\varepsilon\,\partial_t^2 \mathbf{E}=\nabla(\rho/\varepsilon)+\mu\partial_t\mathbf{J}\) \newline
\(\nabla^2 \mathbf{B}-\mu\varepsilon\,\partial_t^2 \mathbf{B}=-\mu\nabla\times\mathbf{J}\) \\
\hline
源項 &
直接出現 \(\rho,\mathbf{J}\) &
含有 \(\nabla\rho,\ \partial_t\mathbf{J},\ \nabla\times\mathbf{J}\) \\
\hline
解法便利性 &
適合用綠函數 \(\to\) 得到推遲勢 (retarded potentials) &
源項複雜,較難直接求解,通常透過勢間接求解 \\
\hline
規範依賴 &
需要選規範(此處用 Lorenz gauge) &
無規範問題,因 \(\mathbf{E},\mathbf{B}\) 為物理量 \\
\hline
無源極限 &
變成齊次波動方程,解為平面波/球面波 &
同樣退化為齊次波動方程,與圖 (2)(3) 相符 \\
\hline
\end{tabular}
% ===== 頻域慣例 =====
% 使用 e^{-j\omega t},故 \partial_t \to -j\omega, \ \partial_t^2 \to -\omega^2
% 介質關係:\mathbf{D}=\varepsilon\mathbf{E}, \ \mathbf{B}=\mu\mathbf{H}
% (有損介質可用 \varepsilon_c=\varepsilon - j\sigma/\omega)

\section*{頻域形式($e^{-j\omega t}$ 慣例)}

\subsection*{(1) 勢函數的 Helmholtz 方程(Lorenz 規範)}
Lorenz 規範在頻域為
\[
\nabla\!\cdot\!\mathbf{A} \;-\; j\omega\,\mu\varepsilon\,\Phi \;=\; 0.
\]
對應的勢方程變為 Helmholtz 形式:
\begin{align}
\left(\nabla^2 + k^2\right)\,\Phi(\mathbf{r},\omega) &= -\,\frac{\rho(\mathbf{r},\omega)}{\varepsilon}, \label{eq:Phi_helm}\\[4pt]
\left(\nabla^2 + k^2\right)\,\mathbf{A}(\mathbf{r},\omega) &= -\,\mu\,\mathbf{J}(\mathbf{r},\omega), \label{eq:A_helm}
\end{align}
其中
\[
k^2 \;=\; \omega^2\,\mu\,\varepsilon.
\]
若介質有電導率 $\sigma$,常以
\[
\varepsilon_c \;=\; \varepsilon \;-\; \frac{j\sigma}{\omega}
\]
取代 $\varepsilon$,則 $k^2=\omega^2\mu\varepsilon_c$(衰減與相位同時包含在 $k$ 中)。

\subsection*{(2) 電場的頻域波動方程(含源的一般式)}
由時間域式
\(
\nabla^2\mathbf{E}-\mu\varepsilon\,\partial_t^2\mathbf{E}
=\nabla(\rho/\varepsilon)+\mu\,\partial_t\mathbf{J}
\)
代入 $\partial_t\to -j\omega$ 可得
\begin{equation}
\boxed{\;
\left(\nabla^2 + k^2\right)\mathbf{E}
\;=\;
\nabla\!\left(\frac{\rho}{\varepsilon}\right)
\;-\; j\omega\,\mu\,\mathbf{J}
\;}.
\label{eq:E_helm}
\end{equation}

\subsection*{(3) 磁場的頻域波動方程(含源的一般式)}
由時間域式
\(
\nabla^2\mathbf{B}-\mu\varepsilon\,\partial_t^2\mathbf{B}
=-\,\mu\,\nabla\times\mathbf{J}
\)
得到
\begin{equation}
\boxed{\;
\left(\nabla^2 + k^2\right)\mathbf{B}
\;=\;
-\,\mu\,\nabla\times\mathbf{J}
\;}.
\label{eq:B_helm}
\end{equation}

\subsection*{(4) 頻域的麥斯威爾方程與連續性方程(供對照)}
在 $e^{-j\omega t}$ 慣例下,
\[
\begin{aligned}
&\nabla\times\mathbf{E} = \;+\; j\omega\,\mathbf{B}, 
&&\qquad \nabla\cdot\mathbf{B}=0,\\
&\nabla\times\mathbf{H} = \mathbf{J} \;-\; j\omega\,\mathbf{D},
&&\qquad \nabla\cdot\mathbf{D}=\rho,\\
&\text{連續性:}\quad \nabla\cdot\mathbf{J} = j\omega\,\rho.
\end{aligned}
\]

\subsection*{(5) 無源與均勻介質的特例}
若 $\rho=\mathbf{J}=0$,則 \eqref{eq:E_helm}, \eqref{eq:B_helm} 退化為
\[
\left(\nabla^2 + k^2\right)\mathbf{E}=0,\qquad
\left(\nabla^2 + k^2\right)\mathbf{B}=0,
\]
為標準的 Helmholtz 方程。平面波解滿足
\(
\mathbf{E},\mathbf{H} \propto e^{-j\mathbf{k}\cdot\mathbf{r}},
\)
其中 $\abs{\mathbf{k}}=k$,相速度 $v_p=\omega/k=1/\sqrt{\mu\varepsilon}$。
\section*{推導}
從麥斯威爾方程與向量勢定義出發:
\begin{align}
\nabla\times\mathbf{E} &= -\,\frac{\partial \mathbf{B}}{\partial t}, \label{eq:Faraday}\\
\mathbf{B} &= \nabla\times\mathbf{A}. \label{eq:Bdef}
\end{align}
將 \eqref{eq:Bdef} 代入 \eqref{eq:Faraday}:
\[
\nabla\times\mathbf{E}
= -\,\frac{\partial}{\partial t}\bigl(\nabla\times\mathbf{A}\bigr)
\;\;\Longrightarrow\;\;
\nabla\times\!\left(\mathbf{E}+\frac{\partial \mathbf{A}}{\partial t}\right)=\mathbf{0}.
\]
因為 $\nabla\times\mathbf{F}=\mathbf{0}$ 的向量場 $\mathbf{F}$ 可寫成某純量函數的負梯度(在單連通區域):
\[
\mathbf{E}+\frac{\partial \mathbf{A}}{\partial t}=-\,\nabla\Phi,
\]
故得到
\[
\boxed{\;\mathbf{E}=-\nabla\Phi-\frac{\partial \mathbf{A}}{\partial t}\;}
\]
這一式將電場分解為\;{\it 靜電部分}\;($-\nabla\Phi$) 與\;{\it 感應部分}\;($-\partial_t\mathbf{A}$)。

\section*{規範自由度(Gauge Freedom)}
勢函數不是唯一的。對任意標量函數 $\chi(\mathbf{r},t)$,變換
\[
\mathbf{A}'=\mathbf{A}+\nabla\chi,
\qquad
\Phi'=\Phi-\frac{\partial\chi}{\partial t}
\]
可保持 $\mathbf{E}$ 與 $\mathbf{B}$ 不變(容易驗證 $\mathbf{B}'=\nabla\times\mathbf{A}'=\mathbf{B}$,
且 $\mathbf{E}'=-\nabla\Phi'-\partial_t\mathbf{A}'=\mathbf{E}$)。
常見選擇如 Lorenz 規範 $\nabla\cdot\mathbf{A}+\mu\varepsilon\,\partial_t\Phi=0$ 或 Coulomb 規範 $\nabla\cdot\mathbf{A}=0$,
用來簡化方程的耦合。

\section*{直觀註解}
\(-\nabla\Phi\) 來自電荷分佈(靜電勢),
\(-\partial_t\mathbf{A}\) 對應隨時間變化的磁場所誘發的感應電場(法拉第定律)。
二者共同構成一般情況下的電場。

\section*{1.~數學上的角色}

定義:
\[
\mathbf{B} = \nabla \times \mathbf{A}.
\]
因為
\(\nabla\cdot \mathbf{B} = 0\),必定存在這樣的向量勢 \(\mathbf{A}\)。

再引入純量勢 \(\Phi\),我們也能寫出
\[
\mathbf{E} = -\nabla \Phi - \frac{\partial \mathbf{A}}{\partial t}.
\]

因此 \((\Phi,\mathbf{A})\) 是場的「勢函數」,能把四個麥斯威爾方程壓縮成兩個波動方程。

\section*{2.~在經典電磁學的物理意義}

在大多數情況下,\(\mathbf{A}\) 被視為「方便的中介」,因為 \(\mathbf{E},\mathbf{B}\) 才是直接可測量的量。

然而 \(\mathbf{A}\) 仍有直觀的物理詮釋:

在靜態情況下(\(\partial_t \mathbf{A} = 0\)),\(\mathbf{A}\) 對應於電流分佈的「分佈型磁效應」。例如:
\[
\mathbf{A}(\mathbf{r}) = \frac{\mu_0}{4\pi} \int \frac{\mathbf{J}(\mathbf{r}')}{|\mathbf{r}-\mathbf{r}'|}\,d^3r',
\]
這形式與電勢
\[
\Phi(\mathbf{r}) = \frac{1}{4\pi\varepsilon_0} \int \frac{\rho(\mathbf{r}')}{|\mathbf{r}-\mathbf{r}'|}\,d^3r'
\]
十分相似。

換句話說,\(\mathbf{A}\) 是「磁場的勢函數」,與電流分佈直接相連。

\section*{向量勢與電流分佈的直接相連}

\subsection*{1. 向量勢的波動方程}

在 Lorenz 規範下,向量勢 $\mathbf{A}$ 滿足
\begin{equation}
\nabla^{2}\mathbf{A} - \mu\varepsilon\,\frac{\partial^{2}\mathbf{A}}{\partial t^{2}}
= -\mu\,\mathbf{J}.
\label{eq:Awave}
\end{equation}

右邊的源項就是電流密度 $\mathbf{J}$ 本身,沒有再經過任何空間或時間導數。

因此 $\mathbf{A}$ 可以直接由電流分佈給出推遲勢解:
\begin{equation}
\mathbf{A}(\mathbf{r},t)
= \frac{\mu}{4\pi}\int \frac{\mathbf{J}\!\left(\mathbf{r}',\,t-\tfrac{|\mathbf{r}-\mathbf{r}'|}{v}\right)}
{|\mathbf{r}-\mathbf{r}'|}\,d^3r', \qquad v=\frac{1}{\sqrt{\mu\varepsilon}}.
\end{equation}

---

\subsection*{2. 場的波動方程比較}

\begin{align}
\nabla^2 \mathbf{E}-\mu\varepsilon\,\frac{\partial^2 \mathbf{E}}{\partial t^2}
&=\nabla\!\left(\frac{\rho}{\varepsilon}\right)+\mu\,\frac{\partial \mathbf{J}}{\partial t},
\\[6pt]
\nabla^2 \mathbf{B}-\mu\varepsilon\,\frac{\partial^2 \mathbf{B}}{\partial t^2}
&=-\mu\,\nabla\times\mathbf{J}.
\end{align}

這裡電場 $\mathbf{E}$ 與磁場 $\mathbf{B}$ 的源項包含了
\(\nabla\rho\)、\(\partial_t \mathbf{J}\)、\(\nabla\times\mathbf{J}\) 等導數,
因此不是單純的 $\rho,\mathbf{J}$。

---

\subsection*{3. 比較表}

\renewcommand{\arraystretch}{1.4}
\begin{tabular}{|>{\raggedright}m{4cm}|>{\raggedright}m{5cm}|>{\raggedright\arraybackslash}m{5cm}|}
\hline
 & \textbf{向量勢 $\mathbf{A}$} & \textbf{電場/磁場 $\mathbf{E},\mathbf{B}$} \\
\hline
波動方程右邊的源 &
$-\mu\,\mathbf{J}$(電流本身) &
$\nabla\rho,\ \partial_t \mathbf{J},\ \nabla\times\mathbf{J}$ 等導數 \\
\hline
與電流的關係 &
\emph{直接相連}:知道 $\mathbf{J}$,即可積分得 $\mathbf{A}$ &
\emph{間接相連}:需要經過導數或時間變化才能得到源項 \\
\hline
解法便利性 &
可直接用格林函數求推遲勢 (retarded potentials) &
通常透過勢來解,再由 $\mathbf{E}=-\nabla\Phi-\partial_t\mathbf{A}$、$\mathbf{B}=\nabla\times\mathbf{A}$ 還原 \\
\hline
\end{tabular}

---

\subsection*{4. 意義}
所謂「\textbf{直接相連}」就是指:
\begin{quote}
在數學上,$\mathbf{A}$ 的源項就是電流密度 $\mathbf{J}$,沒有額外的導數。
因此 $\mathbf{A}$ 能夠以一個簡單的推遲積分形式直接由電流分佈決定。
相對地,$\mathbf{E},\mathbf{B}$ 的波動方程則包含 $\rho,\mathbf{J}$ 的導數,與源之間的關係較為間接。
\end{quote}
\section*{1.~方程回顧}
徑向對稱下($A_x=A_x(r)$),Helmholtz 方程化為
\begin{equation}
\frac{1}{r^2}\frac{d}{dr}\!\left(r^2 \frac{dA_x}{dr}\right)+\beta^2 A_x=0.
\label{eq:radial}
\end{equation}

\section*{2.~消去一階項的代換}
令
\[
A_x(r)=\frac{u(r)}{r}.
\]
代入 \eqref{eq:radial} 並化簡可得
\[
\frac{1}{r}\Bigl(u''+\beta^2 u\Bigr)=0
\;\;\Longrightarrow\;\;
u''+\beta^2 u=0.
\]

\section*{3.~求解 $u(r)$ 與還原 $A_x(r)$}
常係數二階常微分方程
\(
u''+\beta^2 u=0
\)
的通解為
\[
u(r)=C_1 e^{-j\beta r}+C_2 e^{+j\beta r}.
\]
故
\[
\boxed{\;
A_x(r)=\frac{C_1 e^{-j\beta r}}{r}+\frac{C_2 e^{+j\beta r}}{r}\;}.
\]

\section*{4.~物理解讀與常見邊界條件}
\begin{itemize}
  \item $e^{-j\beta r}/r$:\emph{向外傳播的球面波}(輻射場),振幅隨 $1/r$ 衰減。
  \item $e^{+j\beta r}/r$:\emph{向內傳播的球面波}(收斂波)。
  \item \textbf{Sommerfeld 輻射條件}(只允許外傳波)時,取 $C_2=0$,得
  \[
  A_x(r)=\frac{C\,e^{-j\beta r}}{r}.
  \]
  \item 在原點 $r\to 0$ 的正則性:$1/r$ 型解在 $r=0$ 發散,表示此通解適用於\emph{源外區域}($r>0$);若含源需改用格林函數或分段解並配合匹配條件。
\end{itemize}
\section*{1.~從柱坐標下的亥姆霍茲方程開始}
在柱坐標 $(\rho,\phi,z)$ 中,假設場量只依賴徑向 $\rho$,則亥姆霍茲方程
\[
\nabla^2 A + \beta^2 A = 0
\]
化為
\[
\frac{1}{\rho}\frac{d}{d\rho}\!\left(\rho \frac{dA}{d\rho}\right) + \beta^2 A = 0.
\]
這就是零階貝索方程 (Bessel equation of order 0)。

\section*{2.~方程形式}
展開得
\[
\frac{d^2 A}{d\rho^2} + \frac{1}{\rho}\frac{dA}{d\rho} + \beta^2 A = 0,
\]
這正是零階 Bessel 方程:
\[
A''(\rho) + \frac{1}{\rho} A'(\rho) + \beta^2 A(\rho) = 0.
\]

\section*{3.~一般解:Bessel 函數}
一般解可寫成
\[
A(\rho) = C_1 J_0(\beta\rho) + C_2 Y_0(\beta\rho),
\]
其中 $J_0, Y_0$ 分別是零階第一、二類 Bessel 函數。

\section*{4.~漸近形式(大 $\rho$ 時)}
當 $\rho \to \infty$,Bessel 函數的漸近式為
\[
J_0(\beta\rho) \sim \sqrt{\tfrac{2}{\pi \beta \rho}} \cos\!\left(\beta\rho - \tfrac{\pi}{4}\right),
\]
\[
Y_0(\beta\rho) \sim \sqrt{\tfrac{2}{\pi \beta \rho}} \sin\!\left(\beta\rho - \tfrac{\pi}{4}\right).
\]

把正弦、餘弦組合成指數形式,可得
\[
A(\rho) \sim \rho^{-1/2}\Big[C_1 e^{-j\beta\rho} + C_2 e^{j\beta\rho}\Big].
\]

\section*{5.~物理意義}
\begin{itemize}
    \item $e^{-j\beta\rho}$:向外傳播的柱面波。
    \item $e^{+j\beta\rho}$:向內傳播的柱面波。
    \item $\rho^{-1/2}$:振幅衰減因子。\\
    對比球面波是 $1/r$ 衰減(能量分散在球面上),柱面波能量分散在圓柱曲面上,故衰減為 $1/\sqrt{\rho}$。
\end{itemize}

\section*{總結}
柱面波的「通解」最嚴謹的形式是
\[
A(\rho) = C_1 J_0(\beta\rho) + C_2 Y_0(\beta\rho).
\]
在遠場近似(大 $\rho$),Bessel 函數漸近式化為 $\rho^{-1/2}$ 乘上指數波,
這就得到
\[
A(\rho) \sim \rho^{-1/2}\Big(C_1 e^{-j\beta\rho} + C_2 e^{j\beta\rho}\Big).
\]
\section{ideal dipole}
We have
\[
\mathbf{E} = \frac{1}{j\omega \varepsilon} \, \nabla \times \mathbf{H}.
\]

For a Hertzian dipole, the magnetic field is
\[
\mathbf{H} = \frac{I \Delta z}{4\pi}
\left( \frac{j\beta}{r} + \frac{1}{r^{2}} \right)
e^{-j\beta r} \, \sin\theta \, \hat{\boldsymbol\phi}.
\]

Since only $H_\phi$ is nonzero, the curl in spherical coordinates yields
\[
E_r = \frac{1}{j\omega\varepsilon}\,\frac{2\cos\theta}{r}\,H_\phi
= \frac{\eta I\Delta z}{4\pi} \, e^{-jkr} \,
2\cos\theta \left( \frac{1}{r^{2}} - \frac{j}{kr^{3}} \right),
\]
\[
E_\theta = -\frac{1}{j\omega\varepsilon}\,
\frac{\sin\theta}{r} \,\frac{\partial}{\partial r}\big(rH_\phi\big)
= \frac{\eta I\Delta z}{4\pi} \, e^{-jkr} \,
\sin\theta \left( \frac{jk}{r} + \frac{1}{r^{2}} - \frac{j}{kr^{3}} \right),
\]
\[
E_\phi = 0,
\]
where $k=\beta$ and
\[
\eta = \sqrt{\frac{\mu}{\varepsilon}}, \qquad 
\frac{1}{j\omega\varepsilon} = \frac{\eta}{jk}.
\]

---

\textbf{Far-field approximation ($kr\gg1$):}
\[
\mathbf{E} \;\approx\; \frac{\eta I\Delta z}{4\pi}\,
\frac{jk\, e^{-jkr}}{r}\,\sin\theta\,\hat{\boldsymbol\theta},
\]
\[
\mathbf{H} \;\approx\; \frac{I\Delta z}{4\pi}\,
\frac{jk\, e^{-jkr}}{r}\,\sin\theta\,\hat{\boldsymbol\phi},
\]
with
\[
\mathbf{E} = \eta \, \hat{\boldsymbol\theta} \times \mathbf{H}.
\]
\section*{Induction vs. Radiation Fields}

\subsection*{1. Induction (Near Field)}
\begin{itemize}
  \item \textbf{Region:} Very close to the source ($r \ll \lambda$).
  \item \textbf{Field behavior:} Fields are mainly \emph{reactive}, i.e.\ energy is stored in the electric and magnetic fields but not radiated away. 
    \begin{itemize}
      \item $\mathbf{E}$ and $\mathbf{H}$ are generally out of phase.
      \item Energy oscillates back and forth between source and field, similar to inductors/capacitors.
    \end{itemize}
  \item \textbf{Dominant component:} 
    \begin{itemize}
      \item Magnetic field dominates near small current loops.
      \item Electric field dominates near short dipoles.
    \end{itemize}
  \item \textbf{Power flow:} No net power carried away, mostly local energy storage.
\end{itemize}

\subsection*{2. Radiation (Far Field)}
\begin{itemize}
  \item \textbf{Region:} Far from the source ($r \gg \lambda$).
  \item \textbf{Field behavior:} Fields are \emph{radiative}, i.e.\ traveling waves.
    \begin{itemize}
      \item $\mathbf{E}$ and $\mathbf{H}$ are in phase and mutually perpendicular:
      \[
      \mathbf{E} \perp \mathbf{H} \perp \hat{r}.
      \]
      \item They form a transverse electromagnetic (TEM) wave.
    \end{itemize}
  \item \textbf{Dominant component:} Both $\mathbf{E}$ and $\mathbf{H}$ contribute equally, related by the wave impedance of free space:
    \[
    \frac{E}{H} = 377~\Omega.
    \]
  \item \textbf{Power flow:} Real power is carried away from the source, spreading out as radiation.
\end{itemize}

\subsection*{3. Transition Zone (Fresnel Region)}
\begin{itemize}
  \item \textbf{Region:} Intermediate distances ($r \sim \lambda$).
  \item \textbf{Behavior:} Mixture of reactive (induction) and radiative characteristics.
  \item \textbf{Applications:} Important in antenna measurement and radar.
\end{itemize}

\subsection*{Summary}
\[
\text{Induction field: near, reactive, energy stored.}
\]
\[
\text{Radiation field: far, propagating, energy carried away.}
\]
\section*{餘弦定理推導}

考慮任意三角形 $ABC$,三邊分別為 $a, b, c$,其中角 $C$ 為夾在邊 $a$ 與 $b$ 之間的角。  
為了方便推導,我們將三角形放置於座標平面上:

\[
A(0,0), \quad B(b,0), \quad C(x,y)
\]

則三邊長為:
\[
AB = b, \quad AC = \sqrt{x^2 + y^2} = a, \quad BC = \sqrt{(x-b)^2 + y^2} = c.
\]

---

\subsection*{1. 由向量定義出發}

向量:
\[
\overrightarrow{CA} = (-x, -y), \quad
\overrightarrow{CB} = (b - x, -y)
\]

根據餘弦定義:
\[
\overrightarrow{CA} \cdot \overrightarrow{CB} = ab\cos C
\]

計算內積:
\[
(-x)(b - x) + (-y)(-y) = x^2 + y^2 - bx
\]
因此:
\[
ab\cos C = x^2 + y^2 - bx
\]

---

\subsection*{2. 由幾何關係求 $c^2$}

邊 $c$ 的平方為:
\[
c^2 = (x - b)^2 + y^2 = x^2 - 2bx + b^2 + y^2
\]
代入 $x^2 + y^2 = a^2$:
\[
c^2 = a^2 + b^2 - 2bx
\]

又由幾何關係可得 $x = a\cos C$,因此:
\[
\boxed{
c^2 = a^2 + b^2 - 2ab\cos C
}
\]

---

\subsection*{3. 幾何意義}

\begin{itemize}
  \item 當 $C = 90^\circ$ 時,$\cos C = 0$,得 $c^2 = a^2 + b^2$,即畢氏定理。
  \item 當 $C < 90^\circ$ 時,$\cos C > 0$,故 $c^2 < a^2 + b^2$。
  \item 當 $C > 90^\circ$ 時,$\cos C < 0$,故 $c^2 > a^2 + b^2$。
\end{itemize}
\section*{Taylor Series}

若函數 $f(x)$ 在 $x=a$ 附近具有各階導數,則其泰勒展開式為:
\[
f(x) = \sum_{n=0}^{\infty} \frac{f^{(n)}(a)}{n!}(x - a)^n
\]

其中 $f^{(n)}(a)$ 表示函數在 $x=a$ 處的第 $n$ 階導數。

---

\subsection*{Maclaurin 展開($a=0$)}

\[
f(x) = f(0) + f'(0)x + \frac{f''(0)}{2!}x^2 + \frac{f^{(3)}(0)}{3!}x^3 + \cdots
\]
\section*{常見函數的泰勒展開}
\section*{Far-Field Approximation of $R$ Using Taylor (Binomial) Expansion}

\subsection*{Step 1. Start from the exact expression}

根據幾何關係:
\[
R = \sqrt{r^2 + z'^2 - 2r z' \cos\theta}
\]
我們希望在遠場條件 $z' \ll r$ 下,將 $R$ 展開成可近似的形式。

---

\subsection*{Step 2. Factor out $r^2$}

\[
R = r \sqrt{1 + \left(\frac{z'}{r}\right)^2 - 2\frac{z'}{r}\cos\theta}
\]
定義一個小量:
\[
x = \left(\frac{z'}{r}\right)^2 - 2\frac{z'}{r}\cos\theta
\]
則:
\[
R = r(1 + x)^{1/2}
\]

---

\subsection*{Step 3. Expand using the Taylor (binomial) series}

根據二項式展開式:
\[
(1+x)^{1/2} \approx 1 + \frac{1}{2}x - \frac{1}{8}x^2 + \cdots
\]
代入得:
\[
R = r\left[1 + \frac{1}{2}x - \frac{1}{8}x^2 + \cdots \right]
\]

---

\subsection*{Step 4. Substitute $x = \left(\dfrac{z'}{r}\right)^2 - 2\dfrac{z'}{r}\cos\theta$}

\[
\begin{aligned}
R &= r\left\{1 + \frac{1}{2}\left[\left(\frac{z'}{r}\right)^2 - 2\frac{z'}{r}\cos\theta\right]
- \frac{1}{8}\left[\left(\frac{z'}{r}\right)^2 - 2\frac{z'}{r}\cos\theta\right]^2 + \cdots \right\} \\
&= r\left\{1 - \frac{z'}{r}\cos\theta + \frac{1}{2}\left(\frac{z'}{r}\right)^2
- \frac{1}{8}\left[\left(\frac{z'}{r}\right)^2 - 2\frac{z'}{r}\cos\theta\right]^2 + \cdots \right\}
\end{aligned}
\]

---

\subsection*{Step 5. Simplify higher-order terms}

展開並重新整理可得:
\[
R = r\left\{1 - \frac{z'}{r}\cos\theta
+ \left(\frac{z'}{r}\right)^2\left[\frac{1}{2} - \frac{\cos^2\theta}{2}\right] + \cdots \right\}
\]

---

\subsection*{Step 6. Far-field approximation}

由於 $z'/r \ll 1$,高階項 $(z'/r)^2$ 可忽略,故:
\[
\boxed{R \approx r - z'\cos\theta}
\]

---

\subsection*{Step 7. Physical interpretation}

\begin{itemize}
    \item $r$:觀察點與原點的距離。
    \item $z'\cos\theta$:天線上微小電流元相對於觀察方向的距離修正量。
\end{itemize}

因此,在計算輻射場的相位項(例如 $e^{-jkR}$)時,可保留主要的相位變化項:
\[
e^{-jkR} \approx e^{-jkr} e^{+jkz'\cos\theta}.
\]

這使得有限偶極天線的積分推導大幅簡化。


\section*{從三維電流密度積分到偶極線電流積分的推導}

由磁向量位的定義式:
\[
A_z = \int \frac{J_z(\mathbf{r}')\, e^{-j\beta R}}{4\pi R}\, dV',
\]
其中
\[
R = |\mathbf{r} - \mathbf{r}'| = \sqrt{r^2 + z'^2 - 2r z'\cos\theta}.
\]

---

\subsection*{(1) 假設為細長偶極天線}

對於一條沿 $z$ 軸的細長導線(半徑遠小於波長),
電流僅沿 $z$ 方向流動,且分佈於 $(x',y')=(0,0)$。
因此電流密度可寫為:
\[
J_z(\mathbf{r}') = I(z')\,\delta(x')\,\delta(y'),
\]
其中 $I(z')$ 為導線上的總電流,
$\delta(\cdot)$ 為 Dirac delta 函數。

---

\subsection*{(2) 將三維體積積分簡化為一維線積分}

因為 $dV' = dx'\,dy'\,dz'$,代入上式得:
\[
A_z = \int \frac{e^{-j\beta R}}{4\pi R} I(z')\,\delta(x')\,\delta(y')\, dx'\,dy'\,dz'.
\]

對 $x'$、$y'$ 積分:
\[
\int \delta(x')\,dx' = 1, \quad
\int \delta(y')\,dy' = 1,
\]
留下 $z'$ 的積分:
\[
A_z = \int \frac{I(z')\, e^{-j\beta R}}{4\pi R}\, dz'.
\]

---

\subsection*{(3) 遠場近似:$R \approx r - z'\cos\theta$}

在觀察點距離遠大於天線長度時 ($r \gg z'$),
可令
\[
R \approx r - z'\cos\theta.
\]
此外,分母的 $R$ 變化極小,
因此可在分母中直接以 $r$ 取代:
\[
\frac{e^{-j\beta R}}{R}
\approx
\frac{e^{-j\beta (r - z'\cos\theta)}}{r}
= \frac{e^{-j\beta r}}{r} e^{j\beta z'\cos\theta}.
\]

---

\subsection*{(4) 代回積分式}

\[
A_z = \int \frac{I(z')\, e^{-j\beta (r - z'\cos\theta)}}{4\pi r}\, dz'
= \frac{e^{-j\beta r}}{4\pi r}
\int I(z')\, e^{j\beta z'\cos\theta}\, dz'.
\]

---

\subsection*{(5) 物理意義}

\begin{itemize}
  \item $I(z')$:導線上的電流分佈。
  \item $\dfrac{e^{-j\beta r}}{4\pi r}$:由原點至觀察點的共同相位與球面衰減項。
  \item $e^{j\beta z'\cos\theta}$:天線上不同位置 $z'$ 對應的相位差。
\end{itemize}

因此整個積分
\[
\int I(z') e^{j\beta z'\cos\theta} dz'
\]
描述了電流分佈的「空間傅立葉變換」,
決定了天線的方向性與輻射場型。

---

\[
\boxed{
A_z = \frac{e^{-j\beta r}}{4\pi r}
\int I(z') e^{j\beta z'\cos\theta} dz'.
}
\]

\end{document}