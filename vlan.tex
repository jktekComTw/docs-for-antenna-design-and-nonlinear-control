\documentclass[12pt,a4paper]{article}
\usepackage[utf8]{inputenc}
\usepackage{xeCJK}
\usepackage{geometry}
\usepackage{listings}
\usepackage{xcolor}
\usepackage{array}
\usepackage{booktabs}
\usepackage{hyperref}

\geometry{margin=2.5cm}
\setCJKmainfont{Noto Sans CJK TC}

% 程式碼樣式設定
\lstset{
    backgroundcolor=\color{gray!10},
    basicstyle=\ttfamily\small,
    breaklines=true,
    frame=single,
    numbers=left,
    numberstyle=\tiny,
    showstringspaces=false,
    language=bash
}

% 自定義 YAML 語言支援
\lstdefinelanguage{yaml}{
    keywords={network, version, ethernets, vlans, id, link, addresses, dhcp4},
    keywordstyle=\color{blue}\bfseries,
    ndkeywords={true, false},
    ndkeywordstyle=\color{red}\bfseries,
    identifierstyle=\color{black},
    sensitive=false,
    comment=[l]{\#},
    commentstyle=\color{gray}\ttfamily,
    stringstyle=\color{orange}\ttfamily,
    morestring=[b]',
    morestring=[b]"
}

\title{單線復用 VLAN 技術實作指南}
\author{網路技術文件}
\date{\today}

\begin{document}

\maketitle

\tableofcontents
\newpage

\section{VLAN 基本概念}

\subsection{定義}
單線復用 VLAN (Virtual Local Area Network) 是指在一條物理網路線上同時傳輸多個虛擬區域網路的技術。

\subsection{技術特點}
\begin{itemize}
    \item 使用 802.1Q 標準協定
    \item 透過 VLAN 標籤 (Tag) 區分不同網路
    \item Trunk 連接方式承載多個 VLAN
    \item 提高網路資源使用效率
\end{itemize}

\section{交換器設定}

\subsection{Cisco 交換器設定}

\begin{lstlisting}[caption=Cisco Trunk 設定]
# 進入介面設定模式
interface FastEthernet0/1

# 設定為 trunk 模式
switchport mode trunk

# 允許特定 VLAN 通過
switchport trunk allowed vlan 10,20,30

# 設定原生 VLAN (untagged)
switchport trunk native vlan 1

# 套用設定
exit
\end{lstlisting}

\subsection{HP/Aruba 交換器設定}

\begin{lstlisting}[caption=HP/Aruba Trunk 設定]
# 設定 trunk
interface ethernet 1/1
tagged vlan 10,20,30
untagged vlan 1
exit
\end{lstlisting}

\section{Linux 系統 VLAN 設定}

\subsection{使用 ip 指令}

\begin{lstlisting}[caption=建立 VLAN 介面]
# 載入 VLAN 模組
sudo modprobe 8021q

# 在 eth0 上建立 VLAN 10
sudo ip link add link eth0 name eth0.10 type vlan id 10
sudo ip link set dev eth0.10 up
sudo ip addr add 192.168.10.1/24 dev eth0.10

# 建立 VLAN 20
sudo ip link add link eth0 name eth0.20 type vlan id 20
sudo ip link set dev eth0.20 up
sudo ip addr add 192.168.20.1/24 dev eth0.20
\end{lstlisting}

\subsection{使用 vconfig (舊方法)}

\begin{lstlisting}[caption=vconfig 設定方法]
# 建立 VLAN 介面
sudo vconfig add eth0 10
sudo vconfig add eth0 20

# 設定 IP 位址
sudo ifconfig eth0.10 192.168.10.1 netmask 255.255.255.0 up
sudo ifconfig eth0.20 192.168.20.1 netmask 255.255.255.0 up
\end{lstlisting}

\section{網路介面設定檔}

\subsection{Ubuntu/Debian Netplan 設定}

\begin{lstlisting}[caption=/etc/netplan/01-netcfg.yaml,language=yaml]
network:
  version: 2
  ethernets:
    eth0:
      dhcp4: false
  vlans:
    vlan10:
      id: 10
      link: eth0
      addresses: [192.168.10.1/24]
    vlan20:
      id: 20
      link: eth0
      addresses: [192.168.20.1/24]
\end{lstlisting}

\subsection{CentOS/RHEL 設定}

\begin{lstlisting}[caption=ifcfg-eth0.10]
DEVICE=eth0.10
BOOTPROTO=static
IPADDR=192.168.10.1
NETMASK=255.255.255.0
VLAN=yes
ONBOOT=yes
\end{lstlisting}

\begin{lstlisting}[caption=ifcfg-eth0.20]
DEVICE=eth0.20
BOOTPROTO=static
IPADDR=192.168.20.1
NETMASK=255.255.255.0
VLAN=yes
ONBOOT=yes
\end{lstlisting}

\section{網路架構範例}

\subsection{典型架構}

\begin{center}
\begin{tabular}{|l|l|}
\hline
\textbf{元件} & \textbf{功能} \\
\hline
路由器 & 提供網際網路連接 \\
\hline
Trunk 線路 (eth0) & 承載所有 VLAN 流量 \\
\hline
Linux 伺服器 & VLAN 終端設備 \\
\hline
VLAN 10 & 管理網路 \\
\hline
VLAN 20 & 使用者網路 \\
\hline
VLAN 30 & DMZ 網路 \\
\hline
\end{tabular}
\end{center}

\section{驗證與監控}

\subsection{VLAN 設定驗證}

\begin{lstlisting}[caption=檢查 VLAN 設定]
# 檢查 VLAN 介面
ip link show | grep vlan
cat /proc/net/vlan/config

# 測試連線
ping -I eth0.10 192.168.10.1
ping -I eth0.20 192.168.20.1

# 檢查路由表
ip route show table all
\end{lstlisting}

\subsection{流量監控}

\begin{lstlisting}[caption=監控 VLAN 流量]
# 監控各 VLAN 流量
iftop -i eth0.10
iftop -i eth0.20

# 檢查頻寬使用
vnstat -i eth0.10
vnstat -i eth0.20

# 封包擷取
tcpdump -i eth0 -e  # 查看 VLAN 標籤
\end{lstlisting}

\section{進階設定}

\subsection{VLAN 間路由}

\begin{lstlisting}[caption=啟用 VLAN 間路由]
# 啟用 IP 轉發
echo 1 > /proc/sys/net/ipv4/ip_forward

# 設定 iptables 規則
iptables -A FORWARD -i eth0.10 -o eth0.20 -j ACCEPT
iptables -A FORWARD -i eth0.20 -o eth0.10 -j ACCEPT
\end{lstlisting}

\subsection{服務品質 (QoS) 設定}

\begin{lstlisting}[caption=VLAN QoS 設定]
# 設定 VLAN 優先級
sudo vconfig set_egress_map eth0.10 0 1

# 流量控制
sudo tc qdisc add dev eth0.10 root handle 1: htb default 30
sudo tc class add dev eth0.10 parent 1: classid 1:1 htb rate 100mbit
\end{lstlisting}

\section{OpenWrt 路由器設定}

\subsection{網路設定檔}

\begin{lstlisting}[caption=/etc/config/network]
config interface 'vlan10'
    option ifname 'eth0.10'
    option proto 'static'
    option ipaddr '192.168.10.1'
    option netmask '255.255.255.0'

config interface 'vlan20'
    option ifname 'eth0.20'
    option proto 'static'
    option ipaddr '192.168.20.1'
    option netmask '255.255.255.0'
\end{lstlisting}

\section{疑難排解}

\subsection{常見問題}

\begin{table}[h]
\centering
\begin{tabular}{|p{4cm}|p{8cm}|}
\hline
\textbf{問題} & \textbf{解決方法} \\
\hline
VLAN 不通 & 檢查 8021q 模組是否載入 \\
& \texttt{lsmod | grep 8021q} \\
\hline
無法 ping 通 & 檢查交換器 trunk 設定 \\
& 確認 VLAN ID 設定正確 \\
\hline
效能問題 & 使用 \texttt{ethtool} 檢查網卡設定 \\
& 檢查 CPU 使用率 \\
\hline
\end{tabular}
\end{table}

\subsection{除錯指令}

\begin{lstlisting}[caption=VLAN 除錯]
# 檢查 VLAN 模組
lsmod | grep 8021q

# 檢查介面狀態
ethtool eth0

# 監控封包
tcpdump -i eth0 -e vlan

# 檢查路由
route -n
ip route show
\end{lstlisting}

\section{效能優化}

\subsection{網路介面優化}

\begin{lstlisting}[caption=效能調整]
# 調整網路緩衝區
echo 'net.core.rmem_max = 16777216' >> /etc/sysctl.conf
echo 'net.core.wmem_max = 16777216' >> /etc/sysctl.conf

# 套用設定
sysctl -p

# 調整網卡設定
ethtool -K eth0 tx off rx off
ethtool -K eth0 tso off gso off
\end{lstlisting}

\section{安全考量}

\subsection{VLAN 安全設定}

\begin{itemize}
    \item 設定適當的防火牆規則
    \item 限制 VLAN 間通訊
    \item 定期更新交換器韌體
    \item 監控異常流量
\end{itemize}

\begin{lstlisting}[caption=防火牆規則範例]
# 禁止特定 VLAN 間通訊
iptables -A FORWARD -i eth0.10 -o eth0.30 -j DROP
iptables -A FORWARD -i eth0.30 -o eth0.10 -j DROP

# 記錄可疑流量
iptables -A FORWARD -j LOG --log-prefix "VLAN_TRAFFIC: "
\end{lstlisting}

\section{總結}

單線復用 VLAN 技術能夠有效提升網路資源使用效率,透過適當的設定可以在單一實體線路上建立多個邏輯網路區段。重點包括:

\begin{enumerate}
    \item 正確設定交換器 trunk 連接埠
    \item 在終端設備上建立對應的 VLAN 介面
    \item 適當配置路由和防火牆規則
    \item 定期監控和維護網路效能
\end{enumerate}

\end{document}