\documentclass[12pt,a4paper]{article}
\usepackage{amsmath,amssymb,bm}
\usepackage{geometry}
\usepackage{graphicx}
\usepackage{physics}
\usepackage{xeCJK}
\usepackage{tabularx}
\geometry{margin=1in}
\setCJKmainfont{Noto Serif CJK TC}

\title{電磁輻射原理與能量傳遞 (Electromagnetic Radiation and Energy Propagation)}
\author{}
\date{}

\begin{document}
\maketitle

\section*{1. 在真空中,輻射功率是否會被路徑消耗?}

在\textbf{自由空間 (vacuum)}中,沒有任何導電或吸收介質,因此\textbf{輻射功率不會被路徑消耗}。  
總輻射功率 \( P_{\text{rad}} \) 在空間中保持恆定,只是因為波前展開而導致功率密度減少。

對於\textbf{等向輻射源 (isotropic radiator)}:
\[
S(r) = \frac{P_{\text{rad}}}{4\pi r^2}, \qquad |\mathbf{E}(r)| \propto \frac{1}{r}.
\]
這表示功率密度隨距離平方衰減,並非能量損失,而是\textbf{能量分佈於更大的球面上}。

若有吸收性介質(例如空氣、水或介電材料),則會出現衰減項:
\[
|\mathbf{E}(r)| \propto \frac{e^{-\alpha r}}{r},
\]
其中 \(\alpha\) 為\textbf{attenuation constant (Np/m)}。  
在真空中,\(\alpha = 0\),因此能量不被消耗。

---

\section*{2. 電流分佈如何成為輻射功率?}

時變電流密度 \( \mathbf{J}(\mathbf{r}') \) 會在空間中激發電磁場。  
在頻域下,\textbf{magnetic vector potential} 為:
\[
\mathbf{A}(\mathbf{r}) = \frac{\mu_0}{4\pi}
\int_V \mathbf{J}(\mathbf{r}') \frac{e^{-j\beta R}}{R} \, dV', 
\quad R = |\mathbf{r} - \mathbf{r}'|.
\]
於遠場 (far field) 下可近似為:
\[
\mathbf{A}(\mathbf{r}) \approx 
\frac{\mu_0 e^{-j\beta r}}{4\pi r}
\int_V \mathbf{J}(\mathbf{r}') e^{j\beta \hat{\mathbf{r}}\cdot \mathbf{r}'} dV'.
\]
此式顯示遠場的角向分佈與\textbf{電流分佈的傅立葉變換 (Fourier transform)} 有關。  

輻射功率密度由\textbf{Poynting vector}給出:
\[
\mathbf{S} = \frac{1}{2}\Re\{\mathbf{E}\times\mathbf{H}^*\},
\]
並且總輻射功率為:
\[
P_{\text{rad}} = \int_0^{2\pi}\!\!\int_0^\pi
S(r,\theta,\phi)\,r^2\sin\theta\,d\theta\,d\phi.
\]
由此可見,\textbf{電流分佈決定了輻射方向圖與功率輸出}。

---

\section*{3. 在空氣中,輻射功率如何被消耗?}

空氣並非完全無損,而具有微小導電率 \(\sigma \approx 10^{-14}\,\mathrm{S/m}\) 及複介電常數:
\[
\varepsilon = \varepsilon' - j\varepsilon''.
\]
因此,波在空氣中會有非常小的吸收與衰減。  
傳播常數:
\[
\gamma = \alpha + j\beta = j\omega\sqrt{\mu\varepsilon\left(1 - j\frac{\sigma}{\omega\varepsilon}\right)}.
\]
功率密度隨距離衰減為:
\[
P(r) = P_0 e^{-2\alpha r}.
\]

\begin{center}
\renewcommand{\arraystretch}{1.3}
\setlength{\tabcolsep}{6pt}
\begin{tabularx}{\textwidth}{|c|c|X|}
\hline
\textbf{情況 (Case)} & \textbf{介質 (Medium)} & \textbf{功率變化 (Power Behavior)} \\
\hline
理想真空 (Vacuum) & Lossless & 功率完全守恆,只因球面擴散而密度降低。 \\
\hline
乾燥空氣 (Dry Air, <10 GHz) & 微弱導電 & 幾乎無耗損,可視為真空近似。 \\
\hline
潮濕空氣或毫米波區 (Humid Air / mmWave) & 含氧氣與水氣吸收 & 發生分子吸收,部分能量轉為熱能。 \\
\hline
霧或雨 (Fog / Rain) & 散射 + 吸收 & 顯著衰減,功率明顯減少。 \\
\hline
\end{tabularx}
\end{center}

在一般通訊頻段(1–10 GHz),空氣吸收極小,
因此可認為\textbf{radiation power 幾乎不被空氣消耗}。

---

\section*{4. 電流如何變成輻射 (How Current Becomes Radiation)}

\subsection*{(1) 從 Maxwell’s Equations 出發}

時變電流 \( \mathbf{J}(\mathbf{r},t) \) 會產生變化的磁場與電場:
\[
\nabla \times \mathbf{H} = \mathbf{J} + \frac{\partial \mathbf{D}}{\partial t}, 
\qquad
\nabla \times \mathbf{E} = -\frac{\partial \mathbf{B}}{\partial t}.
\]
變化的磁場誘發電場,變化的電場又產生新的磁場,形成\textbf{自我維持的傳播鏈 (self-sustaining chain)}。

\subsection*{(2) 輻射區域分類}

\begin{center}
\renewcommand{\arraystretch}{1.3}
\setlength{\tabcolsep}{6pt}
\begin{tabularx}{\textwidth}{|c|c|X|}
\hline
\textbf{區域 (Region)} & \textbf{場的性質 (Field Type)} & \textbf{說明 (Description)} \\
\hline
近場 (Reactive Near Field) & \textbf{Reactive field}(儲能型) & 能量在天線與場之間往返振盪,無淨功率流向無限遠。 \\
\hline
遠場 (Radiation Zone / Far Field) & \textbf{Radiating field}(傳播型) & 電場與磁場同相,能量持續向外傳遞。 \\
\hline
\end{tabularx}
\end{center}

\subsection*{(3) 物理過程}
\begin{enumerate}
    \item 射頻電壓 (RF voltage) 使導體中電荷\textbf{加速 (accelerate)}。
    \item 加速電荷產生\textbf{變化的電場與磁場}。
    \item 變化的場依據 Maxwell 方程互相激發。
    \item 電場與磁場形成\textbf{自我傳播的電磁波 (self-propagating wave)}。
    \item 能量脫離天線,無法回收,即為\textbf{電磁輻射 (electromagnetic radiation)}。
\end{enumerate}

\subsection*{(4) 功率觀點}

坡印廷向量:
\[
\mathbf{S} = \frac{1}{2}\Re\{\mathbf{E}\times\mathbf{H}^*\},
\]
描述能量流密度。  
其曲面積分為總輻射功率:
\[
P_{\text{rad}} = \oint_S \mathbf{S}\cdot d\mathbf{A}.
\]

---

\section*{5. 總結 (Summary)}

\begin{center}
\renewcommand{\arraystretch}{1.3}
\setlength{\tabcolsep}{6pt}
\begin{tabularx}{\textwidth}{|c|c|X|}
\hline
步驟 (Step) & 概念 (Concept) & 意涵 (Meaning) \\
\hline
1 & 時變電流 (Time-varying current) & 電荷加速產生變化場 \\
\hline
2 & 變化的場互相耦合 & 根據 Maxwell 方程形成波動 \\
\hline
3 & 場脫離天線形成自我傳播 & 產生輻射 \\
\hline
4 & 坡印廷向量表示能量流 & 功率離開天線傳入空間 \\
\hline
\end{tabularx}
\end{center}

\[
\begin{aligned}
\text{Accelerating charges produce time-varying fields that sustain each other,} \\
\text{and detach from the antenna, forming electromagnetic waves propagating in space.}
\end{aligned}
\]
\section*{Poynting Vector(坡印廷向量)}

\textbf{Poynting vector} 表示電磁場的\textbf{功率流密度(power flow density)},亦即電磁能量通過單位面積的速率與方向。

其數學表達式為:
\[
\mathbf{S} = \mathbf{E} \times \mathbf{H}
\]

其中:
\begin{itemize}
    \item $\mathbf{E}$ 為電場向量(electric field vector)
    \item $\mathbf{H}$ 為磁場向量(magnetic field vector)
    \item $\mathbf{S}$ 為\textbf{坡印廷向量(Poynting vector)},代表能量傳遞的方向與強度
\end{itemize}

\end{document}
