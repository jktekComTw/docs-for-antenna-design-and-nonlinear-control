\documentclass{article}
\usepackage{tikz}
\usetikzlibrary{arrows.meta, positioning, shapes.geometric}
\usepackage[utf8]{inputenc}
\usepackage[UTF8]{ctex}
\usepackage{pgfplots}
\pgfplotsset{compat=1.16}
\usetikzlibrary{calc}
\usepackage{amsmath}
\usepackage{amssymb}
\usepackage{array}
\usepackage{booktabs}
\usepackage{enumitem}
\usepackage{xeCJK}
\usepackage{CJKutf8}
\setCJKmainfont{Noto Serif CJK TC}    % 思源宋體繁體版(推薦)
\setCJKsansfont{Microsoft JhengHei}   % 微軟正黑體
\setCJKmonofont{Noto Sans Mono CJK TC}
% \setCJKmainfont{WenQuanYi Zen Hei}
% % 或者使用其他可用字體:
% % \setCJKmainfont{AR PL Mingti2L Big5}
% % \setCJKmainfont{Noto Sans CJK TC}

% % 設置全形符號支持
% \setCJKsansfont{WenQuanYi Zen Hei}
% \setCJKmonofont{WenQuanYi Zen Hei}

\title{Self-Injection-Locked (SIL) Oscillator Analysis}
\author{carlos ma}
\date{\today}
\usepackage[margin=1in]{geometry}
% \usepackage{tikz}
\usetikzlibrary{arrows.meta,positioning}
\usepackage{caption}
\begin{document}
\maketitle
\section*{訊號合成與還原分析}

假設原始基帶訊號為
\[
\mathbf{x}(t)=
\begin{bmatrix}
I(t)\\[4pt]
Q(t)
\end{bmatrix}.
\]

經過 Hartley 式的 $+90^{\circ}$ 相移後,將 $Q$ 分量與 $I$ 疊加,可表示為
\[
y(t)=a\,I(t)+b\,Q(t),
\]
其中 $a,b$ 為常數或頻率相關的相位運算子。  
若 $I(t)=\cos\omega t,\ Q(t)=\sin\omega t$,則
\[
y(t)=I(t)+Q_{+\!90^{\circ}}(t)
= \cos\omega t + \cos\omega t
= 2\cos\omega t.
\]
此時原本兩個自由度被壓縮成單一標量,資訊已經遺失。

\subsection*{可逆條件}
將系統視為線性映射
\[
\mathbf{y}(t)=\mathbf{A}\,\mathbf{x}(t),
\]
其中 $\mathbf{y}(t)\in\mathbb{R}^{m}$,$\mathbf{x}(t)\in\mathbb{R}^{2}$。  
若只有一個輸出 ($m=1$),則無法從 $y(t)$ 唯一決定 $I(t),Q(t)$。  
若有兩個線性獨立的觀測:
\[
\begin{bmatrix}y_1\\y_2\end{bmatrix}
=\mathbf{A}\,
\begin{bmatrix}I\\Q\end{bmatrix},
\qquad
\mathbf{A}\in\mathbb{R}^{2\times 2},
\]
且 $\det(\mathbf{A})\neq 0$,則可以求逆:
\[
\begin{bmatrix}I\\Q\end{bmatrix}
=\mathbf{A}^{-1}
\begin{bmatrix}y_1\\y_2\end{bmatrix}.
\]

\subsection*{Hartley 雙觀測特例}
若有
\[
y_1=I+Q_{+90^{\circ}}, \qquad
y_2=I+Q_{-90^{\circ}},
\]
可寫成矩陣:
\[
\begin{bmatrix}y_1\\y_2\end{bmatrix}
=
\begin{bmatrix}
1 & h_{+}\\[4pt]
1 & h_{-}
\end{bmatrix}
\begin{bmatrix}I\\Q\end{bmatrix},
\]
其中 $h_{\pm}$ 為 $Q$ 通道相移的係數。若 $h_{+}\neq h_{-}$,矩陣可逆,反解為
\[
\begin{bmatrix}I\\Q\end{bmatrix}
=\frac{1}{h_{-}-h_{+}}
\begin{bmatrix}
h_{-} & -h_{+}\\[4pt]
-1 & 1
\end{bmatrix}
\begin{bmatrix}y_1\\y_2\end{bmatrix}.
\]

\subsection*{結論}
若僅有單一路輸出 $y(t)=I(t)+Q_{+\!90^{\circ}}(t)$,一般情況下無法還原 $I(t),Q(t)$。  
要還原必須:
\begin{itemize}
\item 取得第二個線性獨立的觀測 $y_2(t)$,並做矩陣反演;或
\item 具備先驗資訊(例如已知 $I,Q$ 關係、訓練序列、稀疏假設),利用參數估計或盲分離。
\end{itemize}


\begin{tikzpicture}[node distance=12mm, auto, >=Stealth]

  % Nodes
  \node (I)      [draw, rectangle] {I(t)};
  \node (Q)      [draw, rectangle, right=10mm of I] {Q(t)};
  \node (phase)  [draw, rectangle, right=10mm of Q] {+90$^\circ$ (Hartley)};
  \node (sum)    [draw, circle, right=18mm of phase] {+};
  \node (y1)     [draw, rectangle, right=18mm of sum] {y$_1$(t) = I + Q$_{+90^\circ}$};

  % Optional second branch for invertibility
  \node (phase2) [draw, rectangle, below=10mm of phase] {-90$^\circ$};
  \node (sum2)   [draw, circle, right=18mm of phase2] {+};
  \node (y2)     [draw, rectangle, right=18mm of sum2] {y$_2$(t) = I + Q$_{-90^\circ}$};

  % Inverse block (when A invertible)
  \node (inv)    [draw, rectangle, below=18mm of y1, minimum width=60mm] {Matrix inversion / digital demix: $\begin{bmatrix}I\\Q\end{bmatrix} = \mathbf{A}^{-1}\begin{bmatrix}y_1\\y_2\end{bmatrix}$};

  % Connections
  \draw[->] (I) -- (sum) node[midway, above] {};
  \draw[->] (Q) -- (phase);
  \draw[->] (phase) -- (sum);
  \draw[->] (sum) -- (y1);

  % second branch
  \draw[->] (I) |- (phase2);
  \draw[->] (Q) -- (phase2);
  \draw[->] (phase2) -- (sum2);
  \draw[->] (sum2) -- (y2);

  % arrows to inverter
  \draw[->] (y1) -- (inv.north west) node[midway, left] {};
  \draw[->] (y2) -- (inv.north east) node[midway, right] {};

  % captions
  \node[below=2mm of I, font=\footnotesize] {原始兩路};
  \node[below=2mm of y1, font=\footnotesize] {單一路合成(不可逆)};
  \node[below=2mm of inv, font=\footnotesize] {有兩個獨立觀測時可逆};

\end{tikzpicture}
\section*{天線間距至少需為半波長的原因}

考慮兩個接收天線,接收窄頻訊號
\[
s(t) = e^{j 2 \pi f_c t}
\]
入射角為 $\theta$,兩天線接收訊號之相位差為
\[
\Delta \phi = \frac{2 \pi d \sin \theta}{\lambda}
\]
其中 $d$ 為天線間距,$\lambda$ 為波長。

由於相位是模 $2\pi$ 的量,若
\[
\frac{2 \pi d \sin \theta}{\lambda} > 2\pi
\]
則會發生空間混疊 (spatial aliasing),不同角度對應到相同相位差。為避免此現象,必須滿足
\[
\frac{2 \pi d \sin \theta}{\lambda} \leq \pi
\quad \Rightarrow \quad
d \leq \frac{\lambda}{2}
\]

因此天線間距必須小於或等於半波長,才能唯一地由相位差解出入射角,避免角度模糊與假波瓣 (grating lobes)。

---

\section*{收發天線間隔至少需為半波長的原因}

對於同時收發的系統,若發射天線與接收天線間距太近,則會產生強烈的近場耦合,使得接收端前級放大器 (LNA) 飽和,無法接收遠端回波。其互耦阻抗近似為
\[
Z_{12} \propto \frac{e^{-j k d}}{d},
\quad k = \frac{2\pi}{\lambda}
\]
當 $d = \lambda/2$ 時,相位差為 $\pi$,互耦效應減小,可提高發射與接收的隔離度。

實務上,設計天線間距為 $\lambda/2$ 是一個常見折衷:既能降低自干擾,又不致讓天線系統過大,並可維持方向圖一致性。

\section*{頻率鍵控調變 (FSK)}

頻率鍵控 (Frequency-Shift Keying, FSK) 是一種數位調變方式,
以不同的載波頻率表示不同的比特值。
對於二進位 FSK (BFSK),傳輸訊號可表示為:
\[
s(t) =
\begin{cases}
A \cos(2 \pi f_0 t), & \text{比特值為 0} \\[6pt]
A \cos(2 \pi f_1 t), & \text{比特值為 1}
\end{cases}
\]

其中 $f_0$ 與 $f_1$ 分別為兩個載波頻率,且頻率間隔定義為
$\Delta f = |f_1 - f_0|$。

更一般化的表示法可寫成
\[
s(t) = A \cos \!\Bigg(
2\pi f_c t + 2\pi \Delta f \!\!\int_{-\infty}^{t} m(\tau) d\tau
\Bigg)
\]
其中 $f_c = \frac{f_0+f_1}{2}$ 為中心頻率,
$m(t) \in \{-1,+1\}$ 為數位訊號。
由此可見 FSK 實際上改變的是瞬時頻率,
因此屬於連續相位調變 (Continuous-Phase Modulation, CPM) 的一種。

FSK 頻譜具有兩個主要能量集中區域,對應 $f_0$ 與 $f_1$。
接收端可透過兩個匹配濾波器或能量檢測器,分別量測對應頻率的能量,
並選擇較大者作為輸出比特。

\begin{tikzpicture}
\noindent
% ========================
% 時域波形
% ========================
\draw[->] (0,0) -- (8,0) node[right]{時間 $t$};
\draw[->] (0,-1.5) -- (0,1.5) node[above]{訊號 $s(t)$};

% Bit intervals
\foreach \x in {0,2,4,6} {
  \draw[dashed,gray] (\x,-1.5) -- (\x,1.5);
}

% Label bits
\node at (1,1.3) {比特 1};
\node at (3,1.3) {比特 0};
\node at (5,1.3) {比特 1};
\node at (7,1.3) {比特 1};

% Draw waveform (approximation: sine waves with different frequencies)
\draw[domain=0:2,smooth,variable=\t,samples=50] plot ({\t},{sin(720*\t)});
\draw[domain=2:4,smooth,variable=\t,samples=50] plot ({\t},{sin(360*\t)});
\draw[domain=4:6,smooth,variable=\t,samples=50] plot ({\t},{sin(720*\t)});
\draw[domain=6:8,smooth,variable=\t,samples=50] plot ({\t},{sin(720*\t)});

% ========================
% 頻譜
% ========================
\begin{scope}[shift={(0,-4)}]
\draw[->] (0,0) -- (8,0) node[right]{頻率 $f$};
\draw[->] (0,0) -- (0,2) node[above]{功率};

% Center frequency line
\draw[dashed,gray] (4,0) -- (4,2);
\node[gray] at (4,-0.3) {$f_c$};

% Two spectral lines
\draw[thick,blue] (2,0) -- (2,1.5);
\draw[thick,blue] (6,0) -- (6,1.5);

\node at (2,-0.3) {$f_0$};
\node at (6,-0.3) {$f_1$};
\end{scope}

\end{tikzpicture}
\section*{頻率調變連續波雷達 (FMCW)}

FMCW 雷達傳送頻率隨時間變化的連續波,常見為線性掃頻訊號:
\[
s_\text{tx}(t) = A \cos\!\Bigg(
2\pi \Big[f_c t + \frac{B}{2T} t^2 \Big]
\Bigg), \qquad 0 \leq t < T
\]
其中 $f_c$ 為起始頻率,$B$ 為掃頻頻寬,$T$ 為掃頻週期,掃頻斜率定義為
$k = \frac{B}{T}$。

若目標距離為 $R$,回波延遲時間為
\[
\tau = \frac{2R}{c}
\]
接收訊號為
\[
s_\text{rx}(t) = A_r \cos\!\Bigg(
2\pi \Big[f_c (t-\tau) + \frac{B}{2T} (t-\tau)^2 \Big]
\Bigg)
\]

發射與接收訊號混頻後,通過低通濾波器,可得到拍頻訊號
\[
s_\text{beat}(t) \approx \cos(2\pi f_b t + \phi_0),
\qquad
f_b = k \tau = \frac{B}{T} \cdot \frac{2R}{c}
\]
因此可由拍頻頻率計算距離:
\[
R = \frac{c f_b T}{2B}
\]

若目標有徑向速度 $v$,還會產生多普勒頻移
\[
f_D = \frac{2v}{\lambda}
\]
拍頻中同時含有距離與速度資訊,需利用多個 chirp 並進行二維 FFT (range-Doppler map) 來分離。
\begin{tikzpicture}[scale=1.0]

% ========================
% Transmit signal (chirp)
% ========================
\node at (-0.5,2.3) {\textbf{Tx Chirp}};
\draw[->] (0,2) -- (8,2) node[right]{$t$};
\draw[->] (0,2) -- (0,3.5) node[above]{頻率 $f(t)$};

\draw[thick,blue] (0,2.2) -- (2.5,3.2);
\draw[thick,blue] (2.5,2.2) -- (5,3.2);
\draw[thick,blue] (5,2.2) -- (7.5,3.2);
\draw[dashed] (0,2.2) -- (0,1.8);
\draw[dashed] (2.5,3.2) -- (2.5,1.8);
\node[below] at (1.25,1.8) {掃頻週期 $T$};

% ========================
% Receive signal (delayed)
% ========================
\node at (-0.5,0.9) {\textbf{Rx (Delayed)}};
\draw[->] (0,0.5) -- (8,0.5) node[right]{$t$};
\draw[->] (0,0.5) -- (0,1.5) node[above]{頻率 $f(t-\tau)$};

\draw[thick,red] (0.5,0.7) -- (3,1.7);
\draw[thick,red] (3,0.7) -- (5.5,1.7);
\draw[thick,red] (5.5,0.7) -- (8,1.7);

\draw[dashed] (0.5,0.7) -- (0.5,0.3);
\node[below] at (0.5,0.3) {$\tau$};

% ========================
% Beat signal (manual sine-like curve)
% ========================
\node at (-0.5,-0.8) {\textbf{Beat Signal}};
\draw[->] (0,-1.2) -- (8,-1.2) node[right]{$t$};
\draw[->] (0,-1.2) -- (0,0) node[above]{幅度};

% draw a few cycles manually
\draw[thick,green]
  plot[smooth] coordinates {(0,-1.0) (0.5,-0.6) (1.0,-1.0) (1.5,-1.4) (2.0,-1.0)
                           (2.5,-0.6) (3.0,-1.0) (3.5,-1.4) (4.0,-1.0)
                           (4.5,-0.6) (5.0,-1.0) (5.5,-1.4) (6.0,-1.0)
                           (6.5,-0.6) (7.0,-1.0) (7.5,-1.4) (8.0,-1.0)};
\node[right,green] at (8,-0.6) {$f_b = k \tau$};

\end{tikzpicture}

\begin{tikzpicture}[scale=1.0]

% ========================
% 1. Transmit signal (chirp)
% ========================
\node at (-0.5,2.3) {\textbf{Tx Chirp}};
\draw[->] (0,2) -- (8,2) node[right]{$t$};
\draw[->] (0,2) -- (0,3.5) node[above]{頻率 $f(t)$};

\draw[thick,blue] (0,2.2) -- (2.5,3.2);
\draw[thick,blue] (2.5,2.2) -- (5,3.2);
\draw[thick,blue] (5,2.2) -- (7.5,3.2);
\draw[dashed] (0,2.2) -- (0,1.8);
\draw[dashed] (2.5,3.2) -- (2.5,1.8);
\node[below] at (1.25,1.8) {掃頻週期 $T$};

% ========================
% 2. Receive signal (delayed)
% ========================
\node at (-0.5,0.9) {\textbf{Rx (Delayed)}};
\draw[->] (0,0.5) -- (8,0.5) node[right]{$t$};
\draw[->] (0,0.5) -- (0,1.5) node[above]{頻率 $f(t-\tau)$};

\draw[thick,red] (0.5,0.7) -- (3,1.7);
\draw[thick,red] (3,0.7) -- (5.5,1.7);
\draw[thick,red] (5.5,0.7) -- (8,1.7);

\draw[dashed] (0.5,0.7) -- (0.5,0.3);
\node[below] at (0.5,0.3) {$\tau = \frac{2R}{c}$};

% ========================
% 3. Beat signal
% ========================
\node at (-0.5,-0.8) {\textbf{Beat Signal}};
\draw[->] (0,-1.2) -- (8,-1.2) node[right]{$t$};
\draw[->] (0,-1.2) -- (0,0) node[above]{幅度};

\draw[thick,green]
  plot[smooth] coordinates {(0,-1.0) (0.5,-0.6) (1.0,-1.0) (1.5,-1.4) (2.0,-1.0)
                           (2.5,-0.6) (3.0,-1.0) (3.5,-1.4) (4.0,-1.0)
                           (4.5,-0.6) (5.0,-1.0) (5.5,-1.4) (6.0,-1.0)
                           (6.5,-0.6) (7.0,-1.0) (7.5,-1.4) (8.0,-1.0)};
\node[right,green] at (8,-0.6) {$f_b = k \tau$};
\end{tikzpicture}
\vspace{1cm}  % Add vertical space

\begin{center}
\begin{tikzpicture}[scale=1.0]
% ========================
% 4. Range-Doppler Map
% ========================
\begin{scope}[shift={(9,0)}]
\node at (2.5,4.2) {\textbf{Range-Doppler Map}};  % Move from 3.5 to 4.2
\draw[->] (0,0) -- (5,0) node[right]{Range};
\draw[->] (0,0) -- (0,3) node[above]{Doppler (速度)};

% Draw grid
\draw[step=1cm,gray!30,very thin] (0,0) grid (5,3);

% Mark some targets
\filldraw[blue] (1,1) circle (3pt) node[right]{慢速近距離目標};
\filldraw[red] (3,2.5) circle (3pt) node[right]{高速遠距離目標};
\filldraw[green!70!black] (4,0.5) circle (3pt) node[right]{中速遠距離目標};

\end{scope}

\end{tikzpicture}
\end{center}
假設發射端 $I$ 通道頻率為 $\omega_{IF}$,振幅為 $A_{Tx}$,相位雜訊為 $\phi_n$:

\begin{equation}
S_{Tx,I,IF}(t) = A_{Tx}\cos\!\big(\omega_{IF}t + \phi_n(t)\big)
\tag{2.37}
\end{equation}

接收端帶有目標物體的移動資訊的回波訊號在經過混頻器降頻後為:

\begin{equation}
S_{Rx,IF}(t) \approx A_{Rx} \cos\!\left(\omega_{IF}t - \left(\frac{4\pi R}{\lambda} + \frac{4\pi x(t)}{\lambda}\right) + \phi_n\!\left(t - \frac{2R}{c}\right)\right)
\tag{2.38}
\end{equation}

其中 $A_{Rx}$ 為接收訊號振幅,$R$ 為目標物體與雷達間的距離,$x(t)$ 為目標物體的位移資訊,$\lambda$ 為載波波長,$c$ 為光速。
\section*{Hamming Window}

A \textbf{Hamming window} is a type of window function commonly used in digital signal processing (DSP) to reduce \textit{spectral leakage} when performing the Fast Fourier Transform (FFT) of a signal.

\section*{1. What It Does}

When you take an FFT of a finite-length signal, you are implicitly multiplying the signal by a \textbf{rectangular window} (simply truncating it at the start and end).  
This sharp cutoff introduces discontinuities, which spread energy into other frequencies---a phenomenon called \textbf{spectral leakage}.

A Hamming window tapers the signal smoothly toward zero at the edges, thereby reducing these discontinuities.

\section*{2. Mathematical Definition}

The Hamming window for \(N\) points is defined as:

\[
w[n] = 0.54 - 0.46 \cos\!\left(\frac{2 \pi n}{N-1}\right),
\qquad 0 \leq n \leq N-1
\]

where:
\begin{itemize}[nosep]
    \item \(N\): window length (number of samples)
    \item \(n\): sample index
\end{itemize}

\section*{3. Shape}

\begin{itemize}[nosep]
    \item Starts and ends near \(0.08\) (not exactly zero)
    \item Maximum value is \(1\) at the center
    \item Smooth bell-like shape
\end{itemize}

This shape preserves more of the signal amplitude compared to a Hann window while still reducing leakage.

\section*{4. Effect in Frequency Domain}

\begin{itemize}[nosep]
    \item \textbf{Main lobe:} Slightly wider than a rectangular window, resulting in slightly worse frequency resolution.
    \item \textbf{Side lobes:} Much lower (about \(-42\ \mathrm{dB}\)), providing much better suppression of leakage from nearby strong frequencies.
\end{itemize}

\section*{5. When to Use}

Use a Hamming window when:
\begin{itemize}[nosep]
    \item You care about amplitude accuracy (it introduces less amplitude distortion than a Hann window).
    \item You want a good balance between main-lobe width (frequency resolution) and side-lobe suppression (leakage reduction).
\end{itemize}
\section{QSIL in Radar Systems}

Quadrature Signal Image Leakage (QSIL) is a critical imperfection in radar receivers that employ I/Q downconversion.  
It refers to the residual mirror-frequency component caused by gain/phase imbalance in the I/Q channels.

\subsection{Signal Model}

Assume the baseband radar echo can be represented as:
\[
r(t) = A e^{j(2\pi f_d t + \phi)}
\]
where:
\begin{itemize}
    \item \(A\) is the echo amplitude,
    \item \(f_d\) is the Doppler frequency shift,
    \item \(\phi\) is the initial phase.
\end{itemize}

After imperfect I/Q demodulation with gain imbalance \(G_I, G_Q\) and phase error \(\Delta \phi\), the output becomes:
\[
y(t) = G_I r_I(t) + G_Q e^{j\Delta \phi} r_Q(t)
\]

This can be expressed as:
\[
y(t) = k_1 r(t) + k_2 r^*(t)
\]

where \(k_2 r^*(t)\) is the image component at frequency \(-f_d\).

\subsection{Impact on Radar Processing}

\begin{itemize}
    \item \textbf{Ghost Targets:} Image components create symmetric false targets at \(-f_d\).
    \item \textbf{Sensitivity Loss:} Residual image power raises the noise floor, reducing SNR.
    \item \textbf{Velocity Ambiguity:} In FMCW or CW radar, poor image rejection may cause wrong direction of velocity estimation.
\end{itemize}

\subsection{Image Rejection Ratio (IRR)}

QSIL is usually quantified by the Image Rejection Ratio (IRR):
\[
\mathrm{IRR} = 10 \log_{10} \!\left( \frac{P_{+f_d}}{P_{-f_d}} \right) \quad [\mathrm{dB}]
\]

Typical requirements:
\begin{itemize}
    \item Commercial radar: \( \mathrm{IRR} \geq 30 \ \mathrm{dB} \)
    \item High-performance radar: \( \mathrm{IRR} \geq 40 \sim 50 \ \mathrm{dB} \)
\end{itemize}

\subsection{Compensation Methods}

\begin{itemize}
    \item Hardware calibration of I/Q mixer gain and phase.
    \item Digital compensation (e.g., adaptive filtering, MMSE-based correction).
    \item High-accuracy quadrature LO generation.
\end{itemize}
\section{Quadrature Self-Signal Injection (QSSI)}

Quadrature Self-Signal Injection (QSSI) refers to the leakage of the local oscillator (LO) or transmit signal 
into the receiver path. In radar systems, this leakage is downconverted by the mixer and appears 
as a \textbf{DC component} in baseband, creating a false target at zero range.

\subsection{Signal Model}

Let the LO signal be:
\[
s_\mathrm{LO}(t) = A_\mathrm{LO} e^{j\omega_\mathrm{LO} t}
\]

If a fraction \(k\) of this signal couples into the receiver input, the leakage signal becomes:
\[
s_\mathrm{leak}(t) = k A_\mathrm{LO} e^{j\omega_\mathrm{LO} t}
\]

After mixing with the LO, the baseband result is:
\[
y_\mathrm{DC} = k A_\mathrm{LO} \cdot A_\mathrm{LO}^* = k |A_\mathrm{LO}|^2
\]

which is a constant DC term corresponding to a zero-range ghost target.

\subsection{Impact on Radar Systems}

\begin{itemize}
    \item \textbf{False Targets:} A strong peak appears at zero range in the range profile or range-Doppler map.
    \item \textbf{Dynamic Range Reduction:} The DC offset may saturate the ADC, masking weak echoes.
    \item \textbf{Baseline Drift:} LO phase noise or temperature drift may cause slow variations of this offset.
\end{itemize}

\subsection{Mitigation Techniques}

\begin{itemize}
    \item Improve hardware isolation (antenna separation, Tx/Rx switch, circulator performance).
    \item Apply DC-offset cancellation in the analog or digital domain.
    \item Subtract a background measurement taken without targets (background calibration).
\end{itemize}

\subsection{Comparison with QSIL}

\begin{table}[h!]
\centering
\begin{tabular}{|c|c|c|c|}
\hline
\textbf{Name} & \textbf{Cause} & \textbf{Frequency Location} & \textbf{Effect} \\
\hline
QSIL & I/Q gain/phase imbalance & Image frequency (\(-f_d\)) & Symmetric ghost targets, degraded IRR \\
\hline
QSSI & LO or Tx leakage & DC (0 Hz) & Zero-range ghost target, elevated noise floor \\
\hline
\end{tabular}
\end{table}

\end{document}