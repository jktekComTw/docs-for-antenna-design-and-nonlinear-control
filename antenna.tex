\documentclass{article}
\usepackage{tikz}
\usetikzlibrary{arrows.meta, positioning, shapes.geometric}
\usepackage[utf8]{inputenc}
\usepackage[UTF8]{ctex}
\usepackage{pgfplots}
\pgfplotsset{compat=1.16}
\usetikzlibrary{calc}
\usepackage{amsmath}
\usepackage{amssymb}
\usepackage{array}
\usepackage{booktabs}
\usepackage{enumitem}
\usepackage{subcaption}
\usepackage[unicode]{hyperref}
\pdfstringdefDisableCommands{%
  \def\pi{pi}%
  \def\Omega{Omega}%
  \def\hat#1{#1}%
  \def\dfrac#1#2{#1/#2}%
  \def\int{\text{∫}}%
}
\usepackage{xeCJK}
\usepackage{bm}
\usepackage{CJKutf8}
\setCJKmainfont{Noto Serif CJK TC}    % 思源宋體繁體版(推薦)
\setCJKsansfont{Microsoft JhengHei}   % 微軟正黑體
\setCJKmonofont{Noto Sans Mono CJK TC}
% \setCJKmainfont{WenQuanYi Zen Hei}
% % 或者使用其他可用字體:
% % \setCJKmainfont{AR PL Mingti2L Big5}
% % \setCJKmainfont{Noto Sans CJK TC}

% % 設置全形符號支持
% \setCJKsansfont{WenQuanYi Zen Hei}
% \setCJKmonofont{WenQuanYi Zen Hei}

\title{design of planar antenna}
\author{carlos ma}
\date{\today}
\usepackage[margin=1in]{geometry}
% \usepackage{tikz}
\usetikzlibrary{arrows.meta,positioning}
\usepackage{caption}
\begin{document}
\maketitle
\section*{法拉第電磁感應定律(微分形)}

\subsection*{1. 方程式}
\begin{equation}
\nabla \times \mathbf{E} \;=\; -\,\frac{\partial \mathbf{B}}{\partial t}
\end{equation}

\subsection*{2. 公式拆解}
\begin{itemize}
  \item $\nabla \times \mathbf{E}$ :電場 $\mathbf{E}$ 的旋度,表示電場是否形成渦旋(環流)。
  \item $-\,\dfrac{\partial \mathbf{B}}{\partial t}$ :磁場 $\mathbf{B}$ 隨時間的變化率,前方的負號代表方向抵抗磁通量的變化。
\end{itemize}

\subsection*{3. 物理意義}
若磁場 $\mathbf{B}$ 在某區域隨時間改變,則該區域會產生一個環形的渦電場 $\mathbf{E}$。這個電場不是由靜電荷產生,而是因為磁場變化而感應出來。負號對應於楞次定律,說明感應電場的方向會抵抗磁通量的改變。

\subsection*{4. 與積分形式的關聯}
積分形式為:
\begin{equation}
\oint_{\partial S} \mathbf{E}\cdot d\mathbf{l}
\;=\; -\,\frac{d}{dt}\int_{S}\mathbf{B}\cdot d\mathbf{A}
\end{equation}
左邊為沿封閉迴路的電場環流(感應電動勢),右邊為穿過該迴路面的磁通量變化率。

\subsection*{5. 實際例子}
\begin{itemize}
  \item \textbf{發電機}:線圈在磁場中旋轉,磁通量隨時間變化,產生感應電壓。
  \item \textbf{變壓器}:原線圈中交流電流改變磁場,使次線圈感應出電壓。
  \item \textbf{無線充電}:發射線圈中的交變磁場在接收線圈中產生感應電流。
\end{itemize}
\section*{前提與假設}
在\textbf{自由空間}(無源、均勻、各向同性)中,電荷密度 $\rho=0$、電流密度 $\bm J=0$,介電常數與磁導率為常數 $\varepsilon_0,\mu_0$。四大方程式簡化為
\begin{align}
\nabla \cdot \bm E &= 0, \qquad &(\text{高斯定律-電}) \label{eq:gaussE}\\
\nabla \cdot \bm B &= 0, \qquad &(\text{高斯定律-磁}) \label{eq:gaussB}\\
\nabla \times \bm E &= -\,\frac{\partial \bm B}{\partial t}, \qquad &(\text{法拉第定律}) \label{eq:faraday}\\
\nabla \times \bm B &= \mu_0 \varepsilon_0 \frac{\partial \bm E}{\partial t}. \qquad &(\text{安培--麥斯威爾}) \label{eq:ampere}
\end{align}


\section*{安培--麥斯威爾定律}

\subsection*{1. 微分形式}
\begin{equation}
\nabla \times \bm{B} \;=\; \mu_0 \bm{J} \;+\; \mu_0 \varepsilon_0 \frac{\partial \bm{E}}{\partial t}
\end{equation}

\begin{itemize}
  \item $\nabla \times \bm{B}$ :磁場 $\bm{B}$ 的旋度,代表磁場在空間中是否呈現「環流」分佈。
  \item $\mu_0 \bm{J}$ :傳導電流 $\bm{J}$ 所貢獻的磁場(安培定律的原始部分)。
  \item $\mu_0 \varepsilon_0 \dfrac{\partial \bm{E}}{\partial t}$ :由隨時間變化的電場(位移電流)所貢獻的磁場。
\end{itemize}

\noindent
👉 麥斯威爾在安培定律中補上了「位移電流項」,才使電磁理論自洽,特別是在電容器充放電的情境中。

\subsection*{2. 積分形式}
\begin{equation}
\oint_{\partial S} \bm{B} \cdot d\bm{l}
\;=\;
\mu_0 \int_S \bm{J} \cdot d\bm{A}
\;+\;
\mu_0 \varepsilon_0 \frac{d}{dt} \int_S \bm{E} \cdot d\bm{A}
\end{equation}

\begin{itemize}
  \item 左邊:沿著邊界迴路的磁場環流。
  \item 右邊第一項:穿過面積 $S$ 的傳導電流。
  \item 右邊第二項:面積上電場通量的變化率(位移電流)。
\end{itemize}

\subsection*{3. 物理意義}
磁場的來源不只是真實電流,還包括隨時間變化的電場。

在電容器充電的例子中,雖然極板之間沒有真實電流通過,但變化的電場相當於「虛擬電流」,它同樣會產生磁場。

這樣修正後,能保證電荷守恆(連續性方程)與電磁方程組的數學一致性。

\subsection*{4. 與電磁波的關係}
\begin{itemize}
  \item 法拉第定律:變化的磁場產生電場。
  \item 安培--麥斯威爾定律:變化的電場產生磁場。
\end{itemize}

兩者交互作用,導致電磁波可以在自由空間中自我傳遞。
\section*{$\mu_0 \varepsilon_0$ 的意義}

在安培--麥斯威爾定律裡出現的
\[
\mu_0 \varepsilon_0
\]
其實是兩個基本常數的乘積:

\subsection*{1. 定義}

\begin{itemize}
  \item $\mu_0$ :\textbf{真空磁導率} (vacuum permeability,又叫磁常數)。  
  在 SI 制中定義為
  \[
  \mu_0 \approx 4\pi \times 10^{-7}\; \text{H/m}
  \]
  (亨利/米)。
  
  \item $\varepsilon_0$ :\textbf{真空介電常數} (vacuum permittivity,又叫電常數)。  
  數值為
  \[
  \varepsilon_0 \approx 8.854 \times 10^{-12}\; \text{F/m}
  \]
  (法/米)。
\end{itemize}

\subsection*{2. 乘積的意義}
它們的乘積出現在電磁波方程中:
\[
c \;=\; \frac{1}{\sqrt{\mu_0 \varepsilon_0}}
\]
這裡 $c$ 就是真空中的光速。

\medskip
👉 換句話說,光速其實由電與磁的基本常數決定。這是麥斯威爾電磁理論最重要的發現之一,也說明光是電磁波。
\section*{高斯定律(磁)}

\subsection*{微分形式}
\begin{equation}
\nabla \cdot \mathbf{B} = 0
\end{equation}

\subsection*{積分形式}
\begin{equation}
\oint_{\partial V} \mathbf{B} \cdot d\mathbf{A} = 0
\end{equation}

\subsection*{物理意義}
沒有磁單極;磁力線不會「起止」,而是形成閉合環。

\section{Gauss's law 兩種等價形式}

\paragraph{微分形(局部說法)}
\begin{equation}
\nabla\!\cdot\!\mathbf{E}=\frac{\rho}{\varepsilon_0}.
\end{equation}
在空間中每一點,電場的散度等於該點體電荷密度 $\rho$ 除以真空介電常數 $\varepsilon_0$。散度 $>0$ 表示電場線在此處湧出(源);$<0$ 表示匯入(匯)。

\paragraph{積分形(整體說法)}
\begin{equation}
\oint_{\partial V}\mathbf{E}\cdot d\mathbf{A}
=\frac{1}{\varepsilon_0}\int_V \rho\, dV.
\end{equation}
任一封閉曲面 $\partial V$ 的電通量等於曲面包住的總電荷除以 $\varepsilon_0$。

\paragraph{兩者等價:散度定理}
\begin{equation}
\int_V (\nabla\!\cdot\!\mathbf{E})\,dV
=\oint_{\partial V}\mathbf{E}\cdot d\mathbf{A}
\ \Rightarrow\ 
\nabla\!\cdot\!\mathbf{E}=\rho/\varepsilon_0.
\end{equation}

\section{物理意義與直觀}
\begin{itemize}[leftmargin=1.5em]
\item 電荷是電場線的源與匯:正電荷放出、負電荷吸入。
\item 電通量衡量穿出曲面的淨場線數量;封閉曲面內淨電荷越大,通量越大。
\item 若曲面內淨電荷為 $0$,總電通量必為 $0$(但 $\mathbf{E}$ 不必為 $0$)。
\end{itemize}

\section{與庫侖定律/位勢的關係}
靜電下 $\nabla\times\mathbf{E}=0$,故 $\mathbf{E}=-\nabla V$。代入微分形得卜松方程
\begin{equation}
\nabla^2 V = -\frac{\rho}{\varepsilon_0},
\end{equation}
在無電荷區域為拉普拉斯方程 $\nabla^2 V=0$。
單一點電荷 $q$ 於原點,$\rho=q\delta(\mathbf{r})$,由積分形得
\begin{equation}
\oint \mathbf{E}\cdot d\mathbf{A}=\frac{q}{\varepsilon_0}
\ \Rightarrow\
\mathbf{E}(r)=\frac{1}{4\pi\varepsilon_0}\frac{q}{r^2}\,\hat{\mathbf{r}},
\end{equation}
即庫侖定律。

\section{高對稱範例與高斯面選擇}
選高斯面使 $\mathbf{E}$ 在面上(i)大小恆定,(ii)與面法向平行或垂直,(iii)部分通量為 $0$,以便積分。

\subsection*{(a) 無限大帶電平面(面電荷密度 $\sigma$)}
取兩端在平面兩側的圓柱(藥罐)作高斯面。側面通量 $=0$,上下兩蓋通量相同:
\begin{equation}
EA + EA = \frac{\sigma A}{\varepsilon_0}
\ \Rightarrow\
E=\frac{\sigma}{2\varepsilon_0}.
\end{equation}
方向由正電面朝外,與距離無關。

\subsection*{(b) 無限長直線電荷(線密度 $\lambda$)}
取半徑 $r$、長度 $L$ 的同軸圓柱。端面通量 $=0$;側面通量 $E(2\pi r L)$:
\begin{equation}
E(2\pi r L)=\frac{\lambda L}{\varepsilon_0}
\ \Rightarrow\
E(r)=\frac{\lambda}{2\pi\varepsilon_0 r}.
\end{equation}

\subsection*{(c) 均勻帶電實心球(體密度 $\rho$,半徑 $R$)}
外部 $r\ge R$:$Q=\frac{4}{3}\pi R^3\rho$,
\begin{equation}
E(r)=\frac{1}{4\pi\varepsilon_0}\frac{Q}{r^2}.
\end{equation}
內部 $r<R$:$Q(r)=\frac{4}{3}\pi r^3\rho$,
\begin{equation}
E(r)=\frac{1}{4\pi\varepsilon_0}\frac{Q(r)}{r^2}
=\frac{\rho\, r}{3\varepsilon_0}.
\end{equation}

\subsection*{(d) 帶電球殼(表面電荷 $\sigma$)}
殼外如點電荷:$E=\dfrac{1}{4\pi\varepsilon_0}\dfrac{Q}{r^2}$;殼內 $E=0$。

\section{邊界條件與面電荷的跳躍}
取穿過界面的極薄 pillbox 作高斯面,可得電場法向分量的跳躍:
\begin{equation}
(\mathbf{E}_{\text{外}}-\mathbf{E}_{\text{內}})\cdot\hat{\mathbf{n}}
=\frac{\sigma}{\varepsilon_0},
\end{equation}
其中 $\sigma$ 為界面上的表面電荷密度,$\hat{\mathbf{n}}$ 由內指向外。
(靜電下 $\nabla\times\mathbf{E}=0$,故切向分量連續。)

\section{介質中的高斯定律(位移向量 $\mathbf{D}$)}
將束縛電荷吸收進極化向量 $\mathbf{P}$,定義
\begin{equation}
\mathbf{D}=\varepsilon_0\mathbf{E}+\mathbf{P},
\qquad
\nabla\!\cdot\!\mathbf{D}=\rho_{\text{free}},
\qquad
\oint \mathbf{D}\cdot d\mathbf{A}=Q_{\text{free}}.
\end{equation}
各向同性線性介質下 $\mathbf{D}=\varepsilon\mathbf{E}$($\varepsilon=\varepsilon_r\varepsilon_0$),計算常以「自由電荷」更方便。

\section{常見陷阱}
\begin{itemize}[leftmargin=1.5em]
\item 必須是封閉曲面;開放面不能直接用積分形。
\item 通量 $=0$ 不代表 $\mathbf{E}=0$,可能僅是進出相抵。
\item 對稱性不足時雖然高斯定律仍成立,但難以直接解出 $\mathbf{E}(\mathbf{r})$,需改用位勢/卜松方程。
\item 單位:$\rho\,[\mathrm{C/m^3}]$,$\varepsilon_0\approx 8.854\times 10^{-12}\ \mathrm{F/m}$;電通量單位 $[\mathrm{V\cdot m}]$ 或 $[\mathrm{N\,m^2/C}]$。
\end{itemize}

\section{解題步驟速記}
\begin{enumerate}[leftmargin=1.5em]
\item 判斷對稱性(球/柱/平面)。
\item 選高斯面使 $\mathbf{E}$ 易處理(恆定/平行或垂直)。
\item 寫通量 $\displaystyle \oint \mathbf{E}\cdot d\mathbf{A}$(多半為 $E\times \text{面積}$)。
\item 計算包住電荷(體/面/線積分或幾何體積 $\times$ 密度)。
\item 用 $\Phi_E=Q_{\text{encl}}/\varepsilon_0$ 解出 $E$,並標明方向。
\end{enumerate}
\section*{對電場 $\bm E$ 的波動方程}
對 \eqref{eq:faraday} 兩側取旋度:
\begin{equation}
\nabla \times (\nabla \times \bm E) \;=\; -\,\frac{\partial}{\partial t}\bigl(\nabla \times \bm B\bigr).
\end{equation}
使用向量恆等式 $\nabla \times (\nabla \times \bm E)=\nabla(\nabla\cdot \bm E)-\nabla^2 \bm E$,並代入 \eqref{eq:gaussE} 與 \eqref{eq:ampere} 得
\begin{align}
\nabla(\nabla\cdot \bm E) - \nabla^2 \bm E &= -\,\frac{\partial}{\partial t}\!\left(\mu_0 \varepsilon_0 \frac{\partial \bm E}{\partial t}\right)\\
0 - \nabla^2 \bm E &= -\,\mu_0 \varepsilon_0 \frac{\partial^2 \bm E}{\partial t^2}.
\end{align}
移項可得\textbf{電場的波動方程}:
\begin{equation}
\boxed{\;\nabla^2 \bm E \;-\; \mu_0 \varepsilon_0 \,\frac{\partial^2 \bm E}{\partial t^2} \;=\; 0\; } \label{eq:waveE}
\end{equation}

\section*{對磁場 $\bm B$ 的波動方程}
同理,對 \eqref{eq:ampere} 兩側取旋度:
\begin{equation}
\nabla \times (\nabla \times \bm B) \;=\; \mu_0 \varepsilon_0 \frac{\partial}{\partial t}\bigl(\nabla \times \bm E\bigr).
\end{equation}
使用恆等式並代入 \eqref{eq:gaussB} 與 \eqref{eq:faraday}:
\begin{align}
\nabla(\nabla\cdot \bm B) - \nabla^2 \bm B &= \mu_0 \varepsilon_0 \frac{\partial}{\partial t}\!\left(-\,\frac{\partial \bm B}{\partial t}\right)\\
0 - \nabla^2 \bm B &= -\,\mu_0 \varepsilon_0 \frac{\partial^2 \bm B}{\partial t^2}.
\end{align}
得到\textbf{磁場的波動方程}:
\begin{equation}
\boxed{\;\nabla^2 \bm B \;-\; \mu_0 \varepsilon_0 \,\frac{\partial^2 \bm B}{\partial t^2} \;=\; 0\; } \label{eq:waveB}
\end{equation}

\section*{波速與光速的關係}
比較 \eqref{eq:waveE} 與 \eqref{eq:waveB} 與標準波動方程
\(
\nabla^2 \bm \Psi - \dfrac{1}{v^2}\dfrac{\partial^2 \bm \Psi}{\partial t^2}=0
\)
可讀出自由空間中的相速度
\begin{equation}
v \;=\; \frac{1}{\sqrt{\mu_0 \varepsilon_0}} \;\equiv\; c,
\end{equation}
即\textbf{光速}。因此光是電磁波,且其速度由 \(\mu_0\) 與 \(\varepsilon_0\) 這兩個基本常數決定。

\section*{平面波解與 $\bm E,\bm B$ 關係(選讀)}
設平面波解
\(
\bm E(\bm r,t)=\bm E_0 e^{i(\bm k\cdot \bm r-\omega t)},\;
\bm B(\bm r,t)=\bm B_0 e^{i(\bm k\cdot \bm r-\omega t)}
\)。
代入 \eqref{eq:faraday} 與 \eqref{eq:ampere} 可得
\begin{align}
\bm k \times \bm E_0 &= \omega \bm B_0,\\
\bm k \times \bm B_0 &= -\,\frac{\omega}{c^2}\bm E_0,
\end{align}
並有
\(
\bm k \cdot \bm E_0 = \bm k \cdot \bm B_0 = 0
\)
(橫波)。在自由空間,
\begin{equation}
|\bm E_0| \;=\; c\,|\bm B_0|, \qquad 
\bm E_0 \perp \bm B_0 \perp \bm k.
\end{equation}

\section*{延伸:介質中的波速}
在均勻、各向同性\textbf{介質}中,以常數 \(\varepsilon, \mu\) 取代 \(\varepsilon_0,\mu_0\),同樣推導得到
\begin{equation}
\nabla^2 \bm E - \mu \varepsilon \,\frac{\partial^2 \bm E}{\partial t^2}=0,\qquad
\nabla^2 \bm B - \mu \varepsilon \,\frac{\partial^2 \bm B}{\partial t^2}=0,
\end{equation}
波速
\(
v=1/\sqrt{\mu \varepsilon} = c/\sqrt{\varepsilon_r \mu_r}
\)
,其中 \(\varepsilon_r=\varepsilon/\varepsilon_0\)、\(\mu_r=\mu/\mu_0\)。

\section*{\texorpdfstring{$4\pi$ 核心觀念}}
\textbf{立體角}(solid angle)以球面面積來定義:
\begin{equation}
\Omega = \frac{A}{r^2}, \qquad [\Omega]=\text{sr},
\end{equation}
其中 $A$ 是半徑 $r$ 的球面上一塊對應面積。整個球面的面積為 $4\pi r^2$,因此整個球面對球心所張立體角
\begin{equation}
\Omega_{\text{sphere}}=\frac{4\pi r^2}{r^2}=4\pi\ \text{sr}.
\end{equation}

\medskip
對於\textbf{任意封閉曲面} $S$,若觀測點 $O$ 位於 $S$ 的內部,則由 $O$ 發出的所有射線都會與 $S$ 相交一次,將 $S$ 做\textbf{放射投影}到同心球面上,得到覆蓋整個球面的一張「貼圖」,故其\textbf{總立體角}必為 $4\pi$。若 $O$ 在 $S$ 外部,射線穿入與穿出相抵,\(\Omega_{\text{total}}=0\)。

\section*{與高斯定律的關係}
對點電荷 $q$,$\mathbf{E}=\dfrac{1}{4\pi\varepsilon_0}\dfrac{q}{r^2}\hat{\mathbf{r}}$。沿封閉曲面 $S$ 的通量為
\begin{equation}
\oint_S \mathbf{E}\cdot d\mathbf{A}
= \frac{q}{4\pi\varepsilon_0}\oint_S \frac{\hat{\mathbf{r}}\cdot d\mathbf{A}}{r^2}
= \frac{q}{4\pi\varepsilon_0}\,\underbrace{\oint_S d\Omega}_{=\,4\pi\ \text{(若 $O$ 在 $S$ 內)}} 
= \frac{q}{\varepsilon_0},
\end{equation}
其中 $d\Omega \equiv \dfrac{\hat{\mathbf{r}}\cdot d\mathbf{A}}{r^2}$ 是微小立體角。這正是高斯定律的幾何本質。

\section*{圖示:放射投影與 \texorpdfstring{$4\pi$}{4π}}
\begin{figure}[h!]
  \centering
  %----------------- Subfigure (a) -----------------%
  \begin{subfigure}[b]{0.48\textwidth}
    \centering
    \begin{tikzpicture}[scale=1.05]
      % Coordinates
      \coordinate (O) at (0,0);
      \def\R{2.2}   % sphere radius
      % Sphere (as circle in 2D)
      \draw[thick] (0,0) circle (\R);
      % Arbitrary closed surface S (a blobby loop enclosing O)
      \draw[very thick,rounded corners=8pt]
        (-2.4,0.4) .. controls (-1.6,1.6) and (-0.8,1.2) .. (-0.2,1.6)
        .. controls (0.6,2.0) and (1.8,1.2) .. (2.2,0.2)
        .. controls (2.4,-0.6) and (1.2,-1.4) .. (0.2,-1.2)
        .. controls (-1.2,-1.0) and (-1.8,-0.2) .. (-2.4,0.4);
      \node[anchor=south east] at (-2.0,1.2) {$S$};
      % Rays from O to S and to sphere
      \foreach \ang in {20,60,120,180,240,300}{
        \draw[thin,-{Latex[length=2mm]}] (O) -- ({\R*cos(\ang)},{\R*sin(\ang)});
      }
      % Mark O
      \fill (O) circle (1.5pt);
      \node[below left] at (O) {$O$};
      % Labels
      \node at (0,-2.8) {(a)從 $O$ 做放射投影:$S$ 的外形投到球面上};
    \end{tikzpicture}
  \end{subfigure}
  \hfill
  %----------------- Subfigure (b) -----------------%
  \begin{subfigure}[b]{0.48\textwidth}
    \centering
    \begin{tikzpicture}[scale=1.05]
      \coordinate (O) at (0,0);
      \def\R{2.2}
      % Sphere
      \shade[inner color=white,outer color=gray!20] (0,0) circle (\R);
      \draw[thick] (0,0) circle (\R);
      % A patch on sphere representing dOmega area
      \begin{scope}
        \clip (0,0) circle (\R);
        \draw[fill=gray!40,opacity=0.7]
          plot[smooth cycle] coordinates {(0.9,1.6) (1.4,1.4) (1.6,0.9) (1.2,0.8) (0.8,1.1)};
      \end{scope}
      % Rays delimiting the patch
      \draw[thin,-{Latex[length=2mm]}] (O) -- (1.1,1.1);
      \draw[thin,-{Latex[length=2mm]}] (O) -- (1.5,0.9);
      % Mark O
      \fill (O) circle (1.5pt);
      \node[below left] at (O) {$O$};
      % Labels
      \node at (0,-2.8) {(b)球面上的區塊對應 $d\Omega=\frac{dA}{r^2}$,總和為 $4\pi$};
    \end{tikzpicture}
  \end{subfigure}
  \caption{任意封閉曲面 $S$ 由內點 $O$ 看出去,放射投影覆蓋整個球面,\(\displaystyle \int\!\!d\Omega=4\pi\).}
\end{figure}

\section*{補充:外部觀測點時為何是 $0$?}
若 $O$ 在 $S$ 外部,沿任意射線穿入與穿出各一次,可將 $d\Omega$ 分為正負兩部分(以外法向為準),積分相互抵消,\(\displaystyle \oint_S d\Omega=0\)。這也對應到無電荷包裹時的零通量情形。

\section*{Dirac Delta Function in Spherical Coordinates}

In Cartesian coordinates, the three-dimensional Dirac delta function is
\begin{equation}
\delta(\mathbf{r}) = \delta(x)\,\delta(y)\,\delta(z),
\end{equation}
which satisfies
\begin{equation}
\int_{\mathbb{R}^3} f(\mathbf{r}) \, \delta(\mathbf{r}) \, d^3r = f(0,0,0).
\end{equation}

In spherical coordinates $(r,\theta,\phi)$, the volume element is
\begin{equation}
d^3r = r^2 \sin\theta \, dr \, d\theta \, d\phi.
\end{equation}

To preserve the defining property of the delta function, we require
\begin{equation}
\delta(\mathbf{r}) = \frac{1}{4\pi r^2}\,\delta(r).
\end{equation}

\subsection*{Verification}
Consider
\begin{align}
\int f(\mathbf{r}) \, \delta(\mathbf{r}) \, d^3r 
&= \int_0^\infty \int_0^\pi \int_0^{2\pi} 
   f(r,\theta,\phi)\,\frac{1}{4\pi r^2}\,\delta(r)\, 
   r^2 \sin\theta \, d\phi \, d\theta \, dr \\
&= \frac{1}{4\pi} \int_0^\infty \delta(r) f(r,\theta,\phi)\, dr 
   \int_0^\pi \sin\theta \, d\theta 
   \int_0^{2\pi} d\phi.
\end{align}

The angular integrals yield
\[
\int_0^\pi \sin\theta \, d\theta = 2, 
\qquad \int_0^{2\pi} d\phi = 2\pi,
\]
so the product is $4\pi$. Therefore,
\begin{equation}
\int f(\mathbf{r}) \, \delta(\mathbf{r}) \, d^3r = f(0).
\end{equation}

\subsection*{Result}
Thus, in spherical coordinates,
\begin{equation}
\boxed{ \delta(\mathbf{r}) = \frac{1}{4\pi r^2}\,\delta(r) }.
\end{equation}

For a point charge $q$ at the origin, the charge density is
\begin{equation}
\rho(\mathbf{r}) = q \, \delta(\mathbf{r}) 
= \frac{q}{4\pi r^2}\,\delta(r).
\end{equation}
\subsection*{Application: Divergence of $\hat{r}/r^2$}

A well-known identity in electromagnetism is
\begin{equation}
\nabla \cdot \left( \frac{\hat{r}}{r^2} \right) = 4\pi \, \delta(\mathbf{r}),
\end{equation}
where $\hat{r} = \mathbf{r}/r$ is the radial unit vector.

\subsubsection*{Proof}
For $r > 0$, we can compute in spherical coordinates:
\begin{equation}
\nabla \cdot \left( \frac{\hat{r}}{r^2} \right)
= \frac{1}{r^2} \frac{\partial}{\partial r} \left( r^2 \cdot \frac{1}{r^2} \right)
= \frac{1}{r^2} \frac{\partial}{\partial r}(1) = 0.
\end{equation}

Thus the divergence vanishes everywhere except at the origin $r=0$.  
To find the contribution at the origin, apply the divergence theorem to a ball of radius $R$:
\begin{align}
\int_{|\mathbf{r}|\leq R} \nabla \cdot \left( \frac{\hat{r}}{r^2} \right) \, d^3r
&= \oint_{|\mathbf{r}|=R} \frac{\hat{r}}{r^2} \cdot d\mathbf{A} \\
&= \oint_{|\mathbf{r}|=R} \frac{\hat{r}}{R^2} \cdot (R^2 \sin\theta \, d\theta \, d\phi \, \hat{r}) \\
&= \int_0^{2\pi} \int_0^\pi \sin\theta \, d\theta \, d\phi \\
&= 4\pi.
\end{align}

Since this holds for any radius $R$, the divergence must be a distribution concentrated at the origin with total integral $4\pi$.  
Hence,
\begin{equation}
\nabla \cdot \left( \frac{\hat{r}}{r^2} \right) = 4\pi \, \delta(\mathbf{r}).
\end{equation}

\end{document}